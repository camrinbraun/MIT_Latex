\chapter{Introduction}
\raggedbottom
The upper, sunlight region of the pelagic ocean, or ``euphotic'' zone, is home to microscopic plants, phytoplankton, which in this well-lit environment are able to thrive and photosynthesize. Though individually quite small, the combined net primary production (NPP) of these diverse organisms in the marine system is estimated to be 48.5 Pg of carbon per year, nearly 50\% of global NPP \citep{Longhurst1995, Field1998}. Due to this significant role in the carbon cycle, the identification of the major factors controlling phytoplankton ecology, physiology, and, ultimately, bloom dynamics has been a central problem in the field of biological oceanography for the past century. From physical explanations (Sverdrup's critical depth hypothesis \citep{Sverdrup1953}), to chemical rationale (Redfield ratio \citep{Redfield1958}), to ecological theory (Margalef's mandala \citep{Margalef1978}), the field has been constantly reevaluating evidence to answer the question: What drives phytoplankton production? \par

Since these foundational hypotheses were put forth, significant advancements in the study of ocean primary production have been made both through the continued collection of traditional biological oceanographic datasets (e.g. chlorophyll, nutrients, taxonomic counts) particularly at long-term sampling sites \citep{Karl1996, Steinberg2001,Smith2003, Li1998}, and through technological advancements. Remote sensing from satellites, starting with radiometric with the Coastal Zone Color Scanner (CZCS) in 1978, enabled global-scale estimates of chlorophyll \textit{a} and spawned a new generation of missions to measure ocean color (e.g. SeaWiFs, MODIS Aqua) \citep{McClain2009}. High-throughput approaches such as flow cytometry, whose early use led to the discovery of the most abundant photosynthetic organism on Earth \citep{Chisholm1988}, have now been expand to enable automated measurements along oceanic transects \citep{Swalwell2011, Ribalet2015} and over time series \citep{Olson2003}.  And, finally, the integration of molecular techniques into the study of marine systems, enabling the discovery of previously unknown diversity \citep{Lopez-Garcia2001}, 
