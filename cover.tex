\title{Movement and oceanographic association of large pelagic fishes in the North Atlantic Ocean}

%\title{Defining ecological and physiological traits of phytoplankton with metatranscriptomics}
%\title{Illuminating the ecological and biogeochemical roles of phytoplankton through metatranscriptomics}
\author{Camrin Donald Braun}
\prevdegrees{M.S., King Abdullah University of Science and Technology (2013) \\
B.S., The College of Idaho (2011)}
% If you wish to list your previous degrees on the cover page, use the 
% previous degrees command:
%       \prevdegrees{A.A., Harvard University (1985)}
% You can use the \\ command to list multiple previous degrees
%       \prevdegrees{B.S., University of California (1978) \\
%                    S.M., Massachusetts Institute of Technology (1981)}
\department{Joint Program in Applied Ocean Science \& Engineering\\Massachusetts Institute of Technology\\ \& Woods Hole Oceanographic Institution}

% If the thesis is for two degrees simultaneously, list them both
% separated by \and like this:
% \degree{Doctor of Philosophy \and Master of Science}
\degree{Doctor of Philosophy}

% As of the 2007-08 academic year, valid degree months are September, 
% February, or June.  The default is June.
\degreemonth{September}
\degreeyear{2018}
\thesisdate{August 10, 2018}

%% By default, the thesis will be copyrighted to MIT.  If you need to copyright
%% the thesis to yourself, just specify the `vi' documentclass option.  If for
%% some reason you want to exactly specify the copyright notice text, you can
%% use the \copyrightnoticetext command.  
%\copyrightnoticetext{\copyright IBM, 1990.  Do not open till Xmas.}

\copyrightnoticetext{\copyright 2018 Camrin Donald Braun. All rights reserved.  
\\ The author hereby grants to MIT and WHOI permission to reproduce and 
to distribute publicly paper and electronic copies of this thesis document 
in whole or in part in any medium now known or hereafter created.}

% If there is more than one supervisor, use the \supervisor command
% once for each.
\supervisor{Simon R. Thorrold}{Senior Scientist\\Woods Hole Oceanographic Institution}

% This is the department committee chairman, not the thesis committee
% chairman.  You should replace this with your Department's Committee
% Chairman.
\chairman{Lauren Mullineaux}{Chair, Joint Committee for Biological Oceanography\\Massachusetts Institute of Technology\\Woods Hole Oceanographic Institution}

% Make the titlepage based on the above information.  If you need
% something special and can't use the standard form, you can specify
% the exact text of the titlepage yourself.  Put it in a titlepage
% environment and leave blank lines where you want vertical space.
% The spaces will be adjusted to fill the entire page.  The dotted
% lines for the signatures are made with the \signature command.
\maketitle

% The abstractpage environment sets up everything on the page except
% the text itself.  The title and other header material are put at the
% top of the page, and the supervisors are listed at the bottom.  A
% new page is begun both before and after.  Of course, an abstract may
% be more than one page itself.  If you need more control over the
% format of the page, you can use the abstract environment, which puts
% the word "Abstract" at the beginning and single spaces its text.

%% You can either \input (*not* \include) your abstract file, or you can put
%% the text of the abstract directly between the \begin{abstractpage} and
%% \end{abstractpage} commands.

% First copy: start a new page, and save the page number.
\cleardoublepage
% Uncomment the next line if you do NOT want a page number on your
% abstract and acknowledgments pages.
% \pagestyle{empty}
\setcounter{savepage}{\thepage}
\begin{abstractpage}
% $Log: abstract.tex,v $
% Revision 1.1  93/05/14  14:56:25  starflt
% Initial revision
% 
% Revision 1.1  90/05/04  10:41:01  lwvanels
% Initial revision
% 
%
%% The text of your abstract and nothing else (other than comments) goes here.
%% It will be single-spaced and the rest of the text that is supposed to go on
%% the abstract page will be generated by the abstract page environment.  This
%% file should be \input (not \include 'd) from cover.tex.
 Marine phytoplankton are central players in the global carbon cycle, responsible for nearly half of global primary production. The identification of the major factors controlling phytoplankton ecology, physiology, and, ultimately, bloom dynamics has been a central problem in the field of biological oceanography for the past century. From physical explanations (Sverdrup's critical depth hypothesis), to chemical rational (Redfield ratio), to ecological theory (Margalef's mandala), the field has been constantly reevaluating evidence to answer the question: What drives phytoplankton blooms? Molecular approaches enable the direct examination of species-specific metabolic profiles in mixed, natural communities, a task which was previously intractable. In this thesis, I developed and applied novel analytical tools and bioinformatic pipelines to characterize the physiological response of phytoplankton at various levels of taxonomic grouping (strain, species, and functional group) to their environment.   \par
An in silico Bayesian statistical approach was designed to identify stable reference genes from high-throughput sequence data for use in RT-qPCR assays or metatranscriptome studies. Using this approach, the first field study, focusing on the species-level, was designed to examine the role of resource partitioning in the coexistence of two closely related diatom species in the same estuarine system. This study demonstrated that co-occurring diatoms in a dynamic coastal marine system have apparent differences in their capacity to use nitrogen and phosphorus, and that these differences may facilitate the diversity of the phytoplankton. The second field study focused on the diatom, haptophyte, and dinoflagellate functional groups and used simulated blooms to characterize the traits that govern the magnitude and timing of phytoplankton blooms in the oligotrophic ocean. The results indicated that blooms form when phytoplankton are released from limitation by resources (nutrients, vitamins, and trace metals) and that the mechanistic basis for the success of one functional group over another may be driven by how efficiently the transcriptome is modulated following a nutrient pulse. The final study looked at the sub-species level, examining the balance of phenotypic plasticity and strain diversity in the success of the cosmopolitan coccolithophore \textit{Emiliania huxleyi}. These data suggest that following perturbation there was a shift in strain composition as well as significant changes in transcript abundance of key biogeochemical genes involved in nutrient acquisition, calcification, and the life stage of the population.\par
Together, these studies demonstrate the breadth of information that can be garnered through the integration of molecular approaches with traditional biological oceanographic surveys, with each illuminating fundamental questions around phytoplankton ecology and bloom formation.




\end{abstractpage}

% Additional copy: start a new page, and reset the page number.  This way,
% the second copy of the abstract is not counted as separate pages.
% Uncomment the next 6 lines if you need two copies of the abstract
% page.
% \setcounter{page}{\thesavepage}
% \begin{abstractpage}
% % $Log: abstract.tex,v $
% Revision 1.1  93/05/14  14:56:25  starflt
% Initial revision
% 
% Revision 1.1  90/05/04  10:41:01  lwvanels
% Initial revision
% 
%
%% The text of your abstract and nothing else (other than comments) goes here.
%% It will be single-spaced and the rest of the text that is supposed to go on
%% the abstract page will be generated by the abstract page environment.  This
%% file should be \input (not \include 'd) from cover.tex.
 Marine phytoplankton are central players in the global carbon cycle, responsible for nearly half of global primary production. The identification of the major factors controlling phytoplankton ecology, physiology, and, ultimately, bloom dynamics has been a central problem in the field of biological oceanography for the past century. From physical explanations (Sverdrup's critical depth hypothesis), to chemical rational (Redfield ratio), to ecological theory (Margalef's mandala), the field has been constantly reevaluating evidence to answer the question: What drives phytoplankton blooms? Molecular approaches enable the direct examination of species-specific metabolic profiles in mixed, natural communities, a task which was previously intractable. In this thesis, I developed and applied novel analytical tools and bioinformatic pipelines to characterize the physiological response of phytoplankton at various levels of taxonomic grouping (strain, species, and functional group) to their environment.   \par
An in silico Bayesian statistical approach was designed to identify stable reference genes from high-throughput sequence data for use in RT-qPCR assays or metatranscriptome studies. Using this approach, the first field study, focusing on the species-level, was designed to examine the role of resource partitioning in the coexistence of two closely related diatom species in the same estuarine system. This study demonstrated that co-occurring diatoms in a dynamic coastal marine system have apparent differences in their capacity to use nitrogen and phosphorus, and that these differences may facilitate the diversity of the phytoplankton. The second field study focused on the diatom, haptophyte, and dinoflagellate functional groups and used simulated blooms to characterize the traits that govern the magnitude and timing of phytoplankton blooms in the oligotrophic ocean. The results indicated that blooms form when phytoplankton are released from limitation by resources (nutrients, vitamins, and trace metals) and that the mechanistic basis for the success of one functional group over another may be driven by how efficiently the transcriptome is modulated following a nutrient pulse. The final study looked at the sub-species level, examining the balance of phenotypic plasticity and strain diversity in the success of the cosmopolitan coccolithophore \textit{Emiliania huxleyi}. These data suggest that following perturbation there was a shift in strain composition as well as significant changes in transcript abundance of key biogeochemical genes involved in nutrient acquisition, calcification, and the life stage of the population.\par
Together, these studies demonstrate the breadth of information that can be garnered through the integration of molecular approaches with traditional biological oceanographic surveys, with each illuminating fundamental questions around phytoplankton ecology and bloom formation.




% \end{abstractpage}

\cleardoublepage
\null\vfill
{\raggedleft\vfill\itshape\Longstack[l]{
  short sweet dedication here. kitty?\bigskip\bigskip\space}\par
}
\cleardoublepage

\section*{\centerline{\LARGE\textsc{Acknowledgments}}}

{\parindent15pt
    %This thesis, like so many of my adventures to date, would not have come to fruition without the support, help, and care of the many incredible people in my life. \par
\singlespace
During my time in the MIT-WHOI Joint Program, I have been supported by the NASA Earth and Space Science Fellowship, the MIT John S. Hennessy Fellowship, the MIT Martin Family Society of Fellows for Sustainability Fellowship, the WHOI Ocean Venture, Grassle, and James Stratton Fellowships and the WHOI Academic Programs Office. This research and its dissemination was supported by funds from the Explorers Club, Rolex, National Geographic, Amazon Web Services, Sigma Xi, MIT Center for International Studies, WHOI Access to the Sea and Coastal Ocean Institute Funds, MIT Graduate Student Council, MIT Student Assistance Fund, WHOI Biology Department, the American Fisheries Society, the WHOI Academic Programs Office and many individual donors. For all of your support, large and small, I am eternally grateful. \par\bigskip

intro \par

advisor \par

committee \par

other institutional folks and groups/depts \par
harriet for templat
boats and data collectors \par

friends, family, etc \par


We thank Willy Hatch and the M/V Machaca, fishermen in the Shark's Eye Tournament (Montauk, NY) and the many volunteers for assistance in field tagging operations. Chris Fischer and OCEARCH sponsored SPOT tag data transmission. We also thank Alice Della Penna and Chi Hin Lam for help in manuscript development. This work was funded by awards to C. Braun from the Martin Family Society of Fellows for Sustainability Fellowship at the Massachusetts Institute of Technology, the Grassle Fellowship and Ocean Venture Fund at the Woods Hole Oceanographic Institution, and the NASA Earth and Space Science Fellowship. This work was made possible by the funders of HYCOM and the Mesoscale Eddy Trajectory Atlas.

We thank D. McGillicuddy, G. Lawson and G. Flierl for helpful discussions while developing this work and C.H. Lam for contributing code. This work was funded by awards to C. Braun from the Martin Family Society of Fellows for Sustainability Fellowship at the Massachusetts Institute of Technology, the Grassle Fellowship and Ocean Venture Fund at the Woods Hole Oceanographic Institution, and the NASA Earth and Space Science Fellowship. Computational support was provided by the AWS Cloud Credits for Research program. Funding for the development of HYCOM has been provided by the National Ocean Partnership Program and the Office of Naval Research. Data assimilative products using HYCOM are funded by the U.S. Navy. Computer time was made available by the DoD High Performance Computing Modernization Program. The output is publicly available at \url{http://hycom.org}. The Mesoscale Eddy Trajectory Atlas products were produced by SSALTO/DUACS and distributed by AVISO+ (\url{http://www.aviso.altimetry.fr/}) with support from CNES, in collaboration with Oregon State University with support from NASA


}
 


