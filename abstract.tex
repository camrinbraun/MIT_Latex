%Abstract
Highly migratory marine fishes support valuable commercial fisheries worldwide but have proven difficult to monitor due to long-distance open ocean migrations and regular excursions to meso- and bathypelagic depths. Meso- and submesoscale features, including fronts, eddies and meanders, comprise the internal weather of the ocean, exciting vertical fluxes and transporting pelagic communities hundreds to thousands of kilometers. Yet, despite the dominance of these features, the biophysical interactions occurring at these scales (< 100 km) remain poorly understood. This leads to a paucity of knowledge on oceanographic drivers of animal movements and hinders management efforts for these species. With ever-improving oceanographic datasets and modeling outputs, we can leverage these tools both to derive better estimates of animal movements and to quantify fish-environment interactions at the oceanic (sub)mesoscale. In this thesis, I developed a novel state-space model framework and applied analytical tools to characterize the biophysical interactions driving animal behavior and species' ecology in the pelagic ocean. A novel, observation-based likelihood framework was combined with an existing but modified Bayesian state-space model to improve geolocation estimates for archival-tagged fishes using oceanographic profile data. The model was incorporated into an open source \texttt{R} package, \texttt{HMMoce} and was validated by comparing outputs to independent, known tracks, demonstrating significant reductions in error as compared to traditional light-level geolocation techniques. Using this approach, I constructed track estimates for a large basking shark tag dataset using a high-resolution oceanographic model and discovered a wide range of movement strategies, including seasonal site-fidelity to the New England shelf and long-range transequatorial migrations. PSAT-tagged swordfish were also tracked using \texttt{HMMoce}, which revealed affinity for thermal front and eddy habitats throughout the North Atlantic that was further corroborated by a fisheries-dependent conventional tag dataset. An additive modeling approach applied to longline catch-per-unit effort data demonstrated the role of biophysical interactions in characterizing areas of higher swordfish catch, including in sea-level anomaly extrema (eddies). The final study used a synergistic analysis of high-resolution, 3D shark movements and satellite observations to quantify the role of mesoscale oceanography in blue and mako shark ecology. This work is demonstrating the importance of (sub)mesoscale features in structuring the pelagic ocean by influencing the movements of apex predators and governing the connectivity between deep scattering layer communities and deep-diving, epipelagic predators. Together, these studies demonstrate the breadth and depth of information that can be garnered through the integration of traditional animal tagging and oceanography research with cutting-edge analytical approaches and high-resolution oceanographic model outputs and remote sensing datasets, the product of which provides a transformative view of the biophysical interactions at the oceanic mesoscale.





