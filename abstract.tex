% $Log: abstract.tex,v $
% Revision 1.1  93/05/14  14:56:25  starflt
% Initial revision
% 
% Revision 1.1  90/05/04  10:41:01  lwvanels
% Initial revision
% 
%
%% The text of your abstract and nothing else (other than comments) goes here.
%% It will be single-spaced and the rest of the text that is supposed to go on
%% the abstract page will be generated by the abstract page environment.  This
%% file should be \input (not \include 'd) from cover.tex.
\textit{Work in progress... }
Marine phytoplankton are central players in the global carbon cycle, responsible for nearly half of global primary production. The identification of the major factors controlling phytoplankton ecology, physiology, and, ultimately, bloom dynamics has been a central problem in the field of biological oceanography for the past century. From physical explanations (Sverdrup's critical depth hypothesis), to chemical rational (Redfield ratio), to ecological theory (Margalef's mandala), the field has been constantly reevaluating evidence to answer the question: What drives phytoplankton blooms? Despite the advancement seen in the field, major knowledge gaps remain as to what factors are most influential to the dynamics of the complex and diverse phytoplankton community. In this thesis, I s


Chapter two describes an \textit{in silico} approach to identifying suitable references genes with stable expression from high throughput sequence data that might then be used RT-qPCR assays or metatranscriptome studies. Chapter three used metatranscriptomic approaches to characterize the nutrient metabolism of two co-occurring diatoms genera, \textit{Skeletonema} and \textit{Thalassiosira} in Narragansett Bay, and provides evidence that suggests resource partitioning may facilitate the vast diversity in the phytoplankton. Chapter four examines the functional group-specific transcriptional traits that underlie bloom dynamics in the oligotrophic ocean. Chapter five investigates the role of strain diversity or a ``seed bank'' in the maintenance of the population of \textit{Emiliania huxleyi}


In spite of their significance to global biogeochemical cycles, there remain, what might be considered by terrestrial ecologists, elementary questions surrounding their physiology and ecology.



%Zooplankton, such as copepods, are highly abundant environmental reservoirs of many bacterial pathogens. Although copepods are known to support diverse and productive bacterial communities, little is understood about whether copepods are affected by bacterial attachment and whether they can regulate these associations through mechanisms such as the innate immune response. This thesis investigates the potential role that copepod physiology may play in regulating Vibrio association and the community structure of its microbiome. To this end, the intrinsic ability of oceanic copepod hosts to transcriptionally respond to mild stressors was first investigated. Specifically, the transcriptional regulation of several heat shock proteins (Hsps), a highly conserved superfamily of molecular chaperones, in the copepod Calanusfinmarchicus was examined and demonstrated that Hsps are a conserved element of the copepod's transcriptional response to stressful conditions and diapause regulation. To then investigate whether copepod hosts respond to and regulate their microbiota, the transcriptomic response of an estuarine copepod Eurytemora affinis to two distinct Vibric species, a free-living strain (V. ordalii 12B09) and a zooplankton specialist (V. sp. F10 9ZB36), was examined with RNA-Seq. Our findings provide evidence that the copepod E. affinis does distinctly recognize and respond to colonizing vibrios via transcriptional regulation of innate immune response elements and transcripts involved in maintaining cuticle integrity. Our work also suggests that association with E. affinis can significantly impact the physiology of Vibrio colonists. Finally, the inter-individual variability of the C.finmarchicus microbiome was examined to identify how specifically and predictably bacterial communities assemble on copepods and whether host physiology influences the bacterial community structure. Our findings suggest that copepods have a predictable "core microbiome" that persists throughout the host's entrance into diapause, a dormancy period characterized by dramatic physiological changes in the host. However, diapausing and active populations harbor distinct flexible microbiomes which may be driven by factors such including the copepod's feeding history, body size, and bacterial interactions. This thesis work highlights the role of copepods as dynamic reservoirs of diverse bacterial communities and implicates copepod host physiology as an important contributor to the activity, abundance, and community structure of its microbiome.
