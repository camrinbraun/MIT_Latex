% $Log: abstract.tex,v $
% Revision 1.1  93/05/14  14:56:25  starflt
% Initial revision
% 
% Revision 1.1  90/05/04  10:41:01  lwvanels
% Initial revision
% 
%
%% The text of your abstract and nothing else (other than comments) goes here.
%% It will be single-spaced and the rest of the text that is supposed to go on
%% the abstract page will be generated by the abstract page environment.  This
%% file should be \input (not \include 'd) from cover.tex.
Marine phytoplankton are central players in the global carbon cycle, responsible for nearly half of global primary production. The identification of the major factors controlling phytoplankton ecology, physiology, and, ultimately, bloom dynamics has been a central problem in the field of biological oceanography for the past century. From physical explanations (Sverdrup's critical depth hypothesis), to chemical rational (Redfield ratio), to ecological theory (Margalef's mandala), the field has been constantly reevaluating evidence to answer the question: What drives phytoplankton blooms?Despite the advancement seen in the field, major knowledge gaps remain as to what factors are most influential to the dynamics of the
complex and diverse phytoplankton community. 


Chapter two describes an \textit{in silico} approach to identifying suitable references genes with stable expression from high throughput sequence data that might then be used RT-qPCR assays or metatranscriptome studies. Chapter three used metatranscriptomic approaches to characterize the nutrient metabolism of two co-occurring diatoms genera, \textit{Skeletonema} and \textit{Thalassiosira} in Narragansett Bay, and provides evidence that suggests resource partitioning may facilitate the vast diversity in the phytoplankton. Chapter four examines the functional group-specific transcriptional traits that underlie bloom dynamics in the oligotrophic ocean. Chapter five investigates the role of strain diversity or a ``seed bank'' in the maintenance of the population of \textit{Emiliania huxleyi}


In spite of their significance to global biogeochemical cycles, there remain, what might be considered by terrestrial ecologists, elementary questions surrounding their physiology and ecology. 
