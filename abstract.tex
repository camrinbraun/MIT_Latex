% $Log: abstract.tex,v $
% Revision 1.1  93/05/14  14:56:25  starflt
% Initial revision
% 
% Revision 1.1  90/05/04  10:41:01  lwvanels
% Initial revision
% 
%
%% The text of your abstract and nothing else (other than comments) goes here.
%% It will be single-spaced and the rest of the text that is supposed to go on
%% the abstract page will be generated by the abstract page environment.  This
%% file should be \input (not \include 'd) from cover.tex.
\textit{Work in progress... }\par
Marine phytoplankton are central players in the global carbon cycle, responsible for nearly half of global primary production. The identification of the major factors controlling phytoplankton ecology, physiology, and, ultimately, bloom dynamics has been a central problem in the field of biological oceanography for the past century. From physical explanations (Sverdrup's critical depth hypothesis), to chemical rational (Redfield ratio), to ecological theory (Margalef's mandala), the field has been constantly reevaluating evidence to answer the question: What drives phytoplankton blooms? Molecular approaches enable the direct examination of species-specific metabolic profiles in mixed, natural communities, a task which was previously intractable. In this thesis, I developed and applied novel analytical tools and bioinformatic pipelines to characterize the intersection of functional (or physiological) and genetic diversity in phytoplankton and its role in ecosystem function.  \par
An in silico Bayesian statistical approach was designed to identify stable reference genes from high-throughput sequence data for use in RT-qPCR assays or metatranscriptome studies (Chapter 2). Using this approach, the first field study was designed to examine the role of resource partitioning in the coexistence of two closely related diatom species in the same estuarine system. This study demonstrated that co-occurring diatoms in a dynamic coastal marine system have apparent differences in their capacity to use nitrogen and phosphorus, and that these differences may facilitate the diversity of the phytoplankton (Chapter 3). The second field study used simulated blooms to characterize the traits that govern the magnitude and timing of phytoplankton blooms in the oligotrophic ocean. The results indicated that blooms form when phytoplankton are released from limitation by resources (nutrients, vitamins, and trace metals) and that the mechanistic basis for the success of one functional group over another may be driven by how efficiently the transcriptome is modulated following a nutrient pulse (Chapter 4). The final study examined the balance of phenotypic plasticity and strain diversity in the success of the cosmopolitan cocolithophore \textit{Emiliania huxleyi}. These data suggest that following perturbation there was a shift in strain composition. Additionally, significant changes in transcript abundance associated with shifts in the ploidy of the population and changes in nutrient physiology and calcification state suggest a strong role of metabolic plasticity (Chapter 5). \par
Together, these studies demonstrate the breadth of information that can be garnered through the integration of molecular approaches with traditional biological oceanographic surveys, with each illuminating fundamental questions around phytoplankton ecology and bloom formation.



