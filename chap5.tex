%\% This is an example first chapter.  You should put chapter/appendix that you
%\% write into a separate file, and add a line \include{yourfilename} to
%\% main.tex, where `yourfilename.tex' is the name of the chapter/appendix file.
%\% You can process specific files by typing their names in at the 
%\% \files=
%\% prompt when you run the file main.tex through LaTeX.

\chapter{Nutrient pulses uniquely drive physiological ecology of cosmopolitan phytoplankton strains}
\raggedbottom






\clearpage





\section{Abstract}
Phytoplankton are well tuned to respond to changing environments, this may happen at the community-level with species or functional group succession, at the species-level through shifts in strain composition, or at the strain-level through alterations to phenotype. Community-level shifts have been well described; however, strain or phenotypic shifts have been more difficult to diagnose in the field. Here, we examined the intersecting roles of metabolic plasticity and strain diversity in the response of natural populations of the common cocolithophore \textit{Emiliania huxleyi} to shifting nutrient regimes in the North Pacific Subtropical Gyre (NPSG). Using a metatranscriptomic approach and leveraging the genome of strain CCMP 1516 and the transcriptomes of four strains of \textit{E. huxleyi} sequenced through the Marine Microbial Eukaryote Transcriptome Sequencing Project (MMETSP), we combined field observations and microcosm studies to track compositional and metabolic responses to nutrient pulses. The \textit{E. huxleyi} transcript pool in the NPSG was dominated by the genome strain. Following the addition of nitrogen, there was a reduction in genome strain transcripts and an increase in transcripts identified as “core,” or common to all strains used in this study. These data suggest that the nutrient pulse caused a shift in the strain composition to a strain not included in our reference set that likely exists at fairly low levels in the NPSG but is adept at responding to nutrient pulses. In addition to the variations in strain diversity observed following nutrient addition, significant changes in transcript abundance were observed and suggest that nitrogen addition induced shifts in the ploidy of the population and changes in nutrient physiology and calcification state. Together, these data underscore the ecological import of the pan-genome of \textit{E. huxleyi}, suggesting that genetic variability within the species complex may be at the heart of its success in a wide variety of marine environments.  
\section{Introduction}
 Potential titles: 
Emiliania huxleyi species complex physiological response and strain variation under changing N and P environments
Pan genome of Emiliania huxleyi maintains species complex in response to nutrient pulses in oligotrophic ocean 
Strain variation facilitates success of cosmopolitan phytoplankton in changing nutrient environments 
Nutrient pulses uniquely drive physiological ecology of cosmopolitan phytoplankton strains
Evidence of strain heterogeneity and modulation in field populations of cosmopolitan phytoplankton in changing nutrient environments
Potential authors: 
Harriet Alexander1,2, Mónica Rouco3, Sheean T. Haley3, and Sonya T. Dyhrman3,* (Sam, Dave?)Still not really sure – would be nice not to have to deal with the back and forth, but we did HD….
1 MIT-WHOI Joint Program in Oceanography/Applied Ocean Science and Engineering, Cambridge, MA 02139, USA
2 Biology Department, Woods Hole Oceanographic Institution, Woods Hole, MA 02543, USA
3 Department of Earth and Environmental Sciences, Lamont-Doherty Earth Observatory, Columbia University, Palisades, NY 10964, USA
Word limit: 5000 words including M/M
Written for submission to ISME Journal





Abstract: 
Phytoplankton are well tuned to respond to changing environments, this may happen at the community-level with functional group succession, at the species-level through shifts in strain composition, or at the strain-level through alterations to phenotype. Community-level shifts have been well described; however, strain or phenotypic shifts have been more difficult to diagnose in the field. Here, we examined the intersecting roles of metabolic plasticity and strain diversity in the response of natural populations of the biogeochemically significant cocolithophore Emiliania huxleyi to shifting nutrient regimes in the North Pacific Subtropical Gyre (NPSG). Using a metatranscriptomic approach and leveraging the genome of strain CCMP 1516 and the transcriptomes of four strains of E. huxleyi sequenced through the Marine Microbial Eukaryote Transcriptome Sequencing Project (MMETSP), we combined field observations and microcosm studies to track compositional and metabolic responses to nutrient pulses. The strain composition of the in situ samples varied little over the sampling period, with CCMP1516, CCMP370 and PLYM219 the most abundant.  Following the addition of nitrogen, both CCMP374 and CCMP379 exhibited dramatic increases. In addition to the variations in strain diversity observed following nutrient addition, significant changes in transcript abundance were observed and suggest that nitrogen addition induced shifts in the ploidy of the population and changes in nutrient physiology and calcification state. Together, these data underscore the ecological import of the pan-genome of E. huxleyi, suggesting that genetic variability within the species complex may be at the heart of its success in a wide variety of marine environments. 
Introduction:
Central to the carbon cycle, marine phytoplankton are estimated to constitute nearly half of global primary productivity (Field et al. 1998). Cocolithophores, a group of marine phytoplankton within the phylum Haptophyta, play a significant role in marine biogeochemical cycles, particularly of carbon and sulfur (Simó 2001). Beyond their contributions to primary production (1-10\% of total marine carbon fixation), cocolithophores are an important source of particulate inorganic carbon in the form of calcite (CaCO3¬¬) constituting about 50\% of calcite deposition to sediments (Poulton et al. 2007). Thus, cocolithophores play a duel role in the cycling of carbon, both in the organic carbon pump, drawing CO2 out of the atmosphere, and the carbonate counter pump, releasing CO2 in the surface layer and ultimately to the atmosphere (Zondervan et al. 2002). The ratio of calcification to carbon fixation has been found to across environemtnal factors such as temperature, salinity, light and nutrients (Paasche 2001; Bollmann and Herrle 2007; Zondervan 2007; Feng et al. 2008). 
Numerically, Emiliania huxleyi is the most abundant cocolithophore species in the modern ocean (Paasche 2001), known for its cosmopolitan distribution in the surface ocean and ability to form large blooms in both eutrophic coastal and oligotrophic open ocean regions (Holligan et al. 1993; Brown and Yoder 1994). Margalef (1978) put forth a hypothesis that such blooms by cocolithophores (K-selected) follow upon the boom-bust dynamics of diatoms (r-selected), which bloom quickly depleting the environment of key nutrients such as nitrogen (N), phosphorus (P), and silica (Si). In these low nutrient environments, it is frequently Emiliania huxleyi, which thought to be the most r-selected of the cocolithophores, that thrives and blooms (Litchman et al. 2006). E. huxleyi is known to be well adapted to such low-nutrient environments with its ability to scavenge nutrients, particularly phosphorus, from organic compounds (Palenik and Henson 1997; Dyhrman and Palenik 2003; Bruhn et al. 2010; Rouco et al. 2013). Understanding the drivers of these large-scale blooms by E. huxleyi, which link oceanic and atmospheric carbon cycles, is increasingly critical, as atmospheric CO2 continues to increase (Meinshausen et al. 2011), altering the carbonate chemistry of the oceans (Raven et al. 2005). 
The potential effects of rising atmospheric CO2 on the production and calcification of these keystone phytoplankton is debated and has been found to vary both across environmental parameters such as nutrient environment (Sciandra et al. 2003; Leonardos and Geider 2005; Rouco et al. 2013) and amongst strains (Riebesell et al. 2000; Iglesias-Rodriguez et al. 2008; Langer et al. 2009). Beyond this response to carbonate chemistry, inter-strain variability has been observed in the ability to grown on various organic substrates (Strom and Bright 2009) and in enzymatic activities (Steinke et al. 1998; Dyhrman and Palenik 2003; Alcolombri et al. 2015). The capacity for this variability was largely revealed through the genome sequencing of many E. huxleyi strains (Read et al. 2013). Historically believed to be a single species, the genome of E. huxleyi, termed a “pan genome”, was found to be highly variable across strains, not only in microsatellite regions, but in gene content, with up to 25\% coding regions variable across species (Read et al. 2013). Such genomic variability has been described in other cosmopolitan species (Kashtan et al. 2014) and may be central to the success of these species in diverse environmental conditions (Biller et al. 2014). While significant diversity as described by non-coding microsatellite loci has been observed in the field (Iglesias-Rodriguez et al. 2006), the variability of gene content as seen across various isolates has not been directly observed in situ. Global surveys of mixed communities have suggested that the known diversity of E. huxleyi may be a cornerstone of its future response to changing ocean conditions (e.g. increased stratification and acidification) (Beaufort et al. 2011).
In spite of the important intersection of strain diversity and physiology in the ecology of E. huxleyi, these dynamics not yet been assessed in the field, being primarily limited to monoculture studies of individual isolates. Applying a metatranscriptomic approach, the relative contribution and activity of different strains of E. huxleyi and their combined physiological signature were tracked both in situ in the North Pacific Subtropical Gyre and under altered nutrient conditions. These data supported the hypothesis by Read et al. (2013) that multiple strains (as evidenced by variable gene sets) exist simultaneously in the environment and that strain composition and global physiology of E. huxleyi are altered under changing nutrient environments. 
Results and Discussion:  	
Total mRNA (>5.0µm) from the surface mixed layer at Station ALOHA was deeply sequenced six times during the summer of 2012, following a Eularian sampling scheme. To perturb the nutrient environment of the community, two identical microcosm experiments were conducted with natural populations ~2 weeks apart. These experiments included a no addition control, a 10\% (v/v) deep (700m) seawater (DSW) treatment to simulated upwelling, and four treatments designed to skew the nitrogen (N) : phosphorus (P) ratio (Supplemental Table Nutrients Added, Supplemental Figure Nutrient Concentrations TF). Sequence reads from the in situ and experimental treatments were conservatively mapped to a custom database comprised of all publicly available transcriptomes in the Marine Microbial Eukaryotic Transcriptome Sequencing Project (MMETSP) (Keeling et al. 2014) as well as to a curated set of available E. huxleyi transcriptomes (Keeling et al. 2014) and the genome (Read et al. 2013) spanning across five strains: CCMP1516, CCMP370, CCMP374, CCMP379, and PLYM219 (Supplemental Table, Strain information).
Variable structuring by nitrogen and trace metal of diatom and haptophyte populations
A analysis of the taxonomic composition of the RNA pool at the species-level across functional groups (Supplemental Figure; Spearman) in the six in situ samples, which were taken over a period of about a month, and the incubation treatments suggests N addition is central to the structuring of both haptophyte and diatom populations in the NPSG. The population structure based on Spearman rank of the two functional groups, however, diverge when Fe and trace metals are added in addition to N. Population structure in both the haptophytes and the diatoms was stable across all in situ samples for both diatoms and haptophytes (Figure 1 A). Following N addition, however, a distinct shift in the population structure of diatoms and haptophytes occurred, with treatments to which N was added (+N, -P, +DSW) clustering separately from those to which no N was added (in situ, control, +P, -N) with UPGMA distance between the two clusters of 0.45 and 0.5 for diatoms and haptophytes, respectively (Figure 1A). In diatoms, clustering was observed between treatments to which trace metals were added in addition to N (-P, +DSW) compared to treatments to which only N was added (UPGMA distance of 0.28) (Figure 1A). This pattern as not observed in haptophytes, where the -P treatments clustered separately from the +N and +DSW treatments (Figure 1A). 
Tracking the most abundant genera within each functional group, however, revealed that while E. huxleyi was consistently the most abundant haptophyte, the dominant diatom changed when N and trace metals were both added. Following nitrogen addition there was an increase in RNA reads associated with Emiliania relative to its abundance in the control, but no other haptophyte became the most abundant (Figure 1C, D). Shifts in the most dominant diatom were observed only with the addition of N and Fe, Si, and trace metals (in the –P and +DSW treatments), with the dominant genera Pseudo-nitzschia replaced by Cylindrotheca in both experiments (Figure 1C, D). The importance of Si and Fe in the structuring of diatom populations is well-documented (Marchetti et al. 2005, 2012), and potentially opens up a niche for E. huxleyi to succeed where diatoms are held back by trace metal or silica deficiency (Morel and Reinfelder 1994; Sunda and Huntsman 1995; Muggli and Harrison 1997). 
Strain variability in changing nutrient environment
Clustering of orthologus proteins was used to isolate the transcript signals from different E. huxleyi strains in the field and examine strain-specific physiological ecology. Across the five strains isolated across the world’s oceans (Figure 2A) there were a total of 132,888 predicted transcripts (Supplemental Table Strain details). Despite the variable isolation locations, the gene content of these strains covered the same major functional classes with similar relative abundances at the broad level of the KOG class (Supplemental Figure KOG distribution). Grouping these transcripts based on predicted protein homology using OrthoMCL (Li et al. 2003) yielded 56,647 distinct orthologous groups. Orthologous groups varied in size and strain representation, with some groups containing proteins from a single strain and some groups having representative proteins from each of the five strains surveyed here (Supplemental Figure Strain groupings). 
There were 5,243 genes identified as shared amongst the five strains here, fewer than the nearly 20,000 “core” genes reported in the comparative genome analysis (Read et al. 2013). To encompass the most comprehensive set of common genes, all proteins identified as ‘core’ by Read et al. (2013) were added to the shared orthologus groups, resulting in 16,914 ‘core’ orthologus groups (Supplemental Figure Core Changes, Figure 2B). This core set accounted for ~80\% of the reads mapping to E. huxleyi in the field and ~70\% in incubations (Supplemental Figure Core Changes). The sum of the five sets of genes unique to each of the strains accounted for ~15\% of all transcripts mapped to E. huxleyi (Figure 2C). Using the strain-specific orthologus groups as a metric for the relative strain composition in the field (or marker of the most active strain), each of the five strains was detected across the month-long sampling period (Figure 2C).  These data are consistent with the finding that single cells of cosmopolitan taxa isolated from the same environment may possess diverse set of genetic alleles and flexible gene sets (Kashtan et al. 2014). These results mirror also mirror microsatellite studies of E. huxleyi that found high diversity at the global scale and many genetic polymorphism between individuals collected in the same sample (Iglesias-Rodriguez et al. 2006). 
The strain composition of the in situ samples varied little over the sampling period. CCMP1516, CCMP370 and PLYM219, had the highest abundances of strain-specific transcripts, while CCMP374, and 379 were less abundant (Figure 2C). Although the dominant haptophyte did not change regardless of nutrient treatment, as did the dominant diatom (Figure 1C, D), shifts in the strain-level composition of the dominant E. huxleyi were observed. In addition to being dominant in the field, CCMP1516 and PLYM219 were the most abundant under treatments where N was not added (Control, -N, +P) (Figure 2D, E). By contrast, both CCMP379 and CCMP374 exhibited dramatic increases (e.g. <1\% to 5\% of reads for CCMP374) following N addition in both experiments (Figure 2D, E). There were some subtle consistent differences, however, with CCMP379 more abundant in –P, which was augmented with iron, trace metals, and vitamins (Figure 2D, 2E). Such enhanced abundance of a particular strain under trace metal repletion is underscored by the heterozyogsity observed in trace metal-associated genes (Read et al. 2013) and trace metal quotas across strains (Sunda and Huntsman 1992). These results suggest that the geochemical environment changes the abundance and activity of different strains in the field, which are similar, though certainly not identical, to the five strains used in this study.  
In addition to the observed shifts in strain abundance following changes in the geochemical environment, the expression of the set of 5,243 shared genes (Supplemental Figure Shared Genes) was examined by strain across the nutrient treatments to statistically validate the apparent shifts in strain abundance and transcriptional response. The 5,243 shared gene set was better annotated than the entire set of orthologus groups, with 73\% of the shared orthologus groups annotated with KOG compared to only 36.9\% for the sum (Supplemental Figure KOG annotation), likely because this set contains essential and well-studied metabolic pathways (e.g. carbon fixation). Focusing on this relatively well-annotated fraction, RSEM (Li and Dewey 2011) was used to estimate the relative contribution of each ‘isoform,’ here defined as the orthologus genes from each of the strains comprising the core orthologus group (Supplemental Figure SharedGeneComp). Principle components analysis (PCA) of the estimated TPM of the 5,243 shared orthologus groups broken down by the five strains across time and experimental condition showed separation of strains and conditions. The first three components of the PCA explained 36.2\% of the variance (Figure 3). The primary separation occurred along the first component (20.5\% of variance), which clearly separated treatments to which N was added from treatments where no N was added (Figure 3C), similar to shifts observed between functional groups following DSW addition (Alexander et al. 2015b). No difference was seen between strains in the first three components in the treatments to which no N was added (Figure 3A,B). The second and third component separated strains in treatments where N was added, with the second component (8.1\%) separating CCMP1516 from the four other strains and the third component (7.6\%) separating CCMP379 from CCMP374 (Figure 3 B). Interestingly, these data appear tied to the unique gene sets, as the unique gene sets of CCMP379 and CCMP374 were observed to increase in abundance following N addition (Figure 2). These data suggest that the metabolic profiles (based only on shared genes) of strains are similar, at least in the context of this very oligotrophic environment, where haptophytes are thought to be limited (Alexander et al. 2015b). Strikingly, the addition of N appears to change the metabolic profile of all strains in a similar way (e.g. enhanced carbon fixation) (Component 1), but differences between the responses of strains in the shared set of genes exist (Components 2 and 3). 
Read et al. postulated that strains are everywhere and that the environment supports—and we show that it holds true for nutrients. 
Read et al. speculated that strains are everywhere envs selects and we show it is true….I think you need some more spin to wrap this up.  This is one of the most significant findings of the paper!!  In Ehux strains shift  - in diatoms – species shift!  Everything is everywhere and the environment selects. Figure 2 rocks! A major driver in this context appears to be a strains unique metabolic response to nutrient addition in an oligotrophic environment.  
5
Conserved physiological response to N addition
The PCA suggests that there is a set of conserved responses to N resupply across the entire E. huxleyi species complex. Taking a conservative approach, the two replicated experiments, which were performed two weeks apart with different initial communities, were considered as biological replicates. Using these biological replicates, the significance of differential abundance was assessed for each of the orthologus groups and individual gene orthologs in each of the amended incubations compared to the no addition control with edgeR. Genes with a two fold increase, FDR < 0.05 were considered to be differentially abundant. Using this approach, only treatments to which nitrate was added (namely, +N, -P, and +DSW) had more than two genes identified as significantly differentially abundant compared to the control. N-amended treatments had between 1,212 and 1,466 orthologus groups with FDR < 0.05, compared to treatments with no N added that had at most 2 genes significantly differentially abundant (Supplemental Figure Manta). This finding, again, suggests that beyond the community being N limited, E. huxleyi likely was constrained by N. More over, the significantly differentially abundant genes and transcripts for three treatments that received N were conserved, with 45\% of differentially abundant orthologs common across the three treatments (Supplemental Figure Venn). Looking to previous literature, genes thought to be associated with nitrogen and phosphorus metabolism (Dyhrman et al. 2006; Rokitta et al. 2014; McKew et al. 2015), calcification and ploidy state (von Dassow et al. 2009; Mackinder et al. 2011; Frada et al. 2012), were blasted (tblastn with an e-value cutoff of 1e-20) against the translated proteins comprising the orthologous groups used in this study. Tracking the expression of the orthologus groups identified, a consistent physiological trend was observed following the addition of N (Figure 4). 
Nitrogen scavenging and assimilation
Broadly speaking, significantly differentially abundant genes following the addition of N could be broken into two groups: 1) genes associated with the response to N-limitation and 2) genes associated with the response to newly available substrates (Figure XA). Genes within this first category of N-limitation responsive were associated both with the scavenging, acquisition, and incorporation of N as well metabolic alterations to energy production and ATP synthesis.
Genes associated with N scavenging and acquisition significantly decreased following the addition of nitrate (2-fold change, FDR < 0.05). Transporters of organic and inorganic nitrogen sources were significantly decreased (FDR < 0.05) following N addition, namely a family of urea transporters (UTP), ammonium transporters (AMT), and nitrate transporters (NRT) (Figure 4; Supplemental Table Genes). These two transporters were found to be significantly increased in the proteome of in N-limited cultures of CCMP1516 (McKew et al. 2015) as well as in the transcriptome of N-limited cultures of CCMP1516 in the haploid stage, though absent from the diploid stage (Rokitta et al. 2014). The ability E. huxleyi to grow on amides and other organic nitrogen has been well documented (Palenik and Henson 1997), yet inter-strain variability has been observed, as this ability is known to be unevenly distributed across different strains of E. huxleyi (Strom and Bright 2009). Of the N metabolism genes surveyed, the largest decrease in abundance was observed in three orthologus groups of amidases and formamidases, which scavenge NH4+ from amides and formamides (Figure 4). Formamidase specific activity and gene expression has been found to be elevated in N-limited cultures of E. huxleyi (Palenik and Henson 1997; Bruhn et al. 2010), but there was substantial inter-strain variability. No specific transcripts were identified as significantly regulated for the amidases, suggesting that no one strain examined here was the primary source for these signals. In each treatment where N was added, three groups of ammonium transporters (AMTs) and a nitrite/formamide transporter significantly increased. As these incubations were run for 7 days in a mixed community, it is likely that there was active nitrogen fixation (Karl et al. 1997)  or remineralization (Casciotti et al. 2008) occurring within the mixed heterotrophic and autotrophic community. 
Beyond the acquisition of N, several markers of changes in N assimilation and energy production were pronounced following the addition of N to the incubations. 
In addition to the decreases in UTP, urease (URE) was significantly decreased following N addition, as observed in N-limited cultures of E. huxleyi (Rokitta et al. 2014), which may serve as a means of accessing N in the form of NH4+ from the ornithine-urea cycle. NH4+ released is then incorporated into biological material through glutamine synthase (GS). Similar to Mckew et al. (2015), we observed a shift between the smaller, GS type II, under low N conditions to the larger GS type III with a higher N requirement, following N addition. Rokitta et al. (2014) in a thorough transcriptomic study noted that in both haploid and diploid stages, E. huxleyi induces a malate:quinone-oxioreductase (MQO) that can bypass malate-dehydrogenase (MDH) in the TCA cycle when respiration and feed electrons directly into the electron transport chain to enable the production of ATP. MQO was significantly decreased (between 5 and 9 log fold change) in both +N and +DSW (Supplemental Figure CarbonNucleotideAA). Additionally, though not significant, MDH increased 6.4 fold under N addition (FDR = 0.36) (Supplemental Figure TCA). The MQO, absent from diatom genomes, and found to be highly expressed both in N-limited culture and in this oligotrophic setting, may be central to the conserved N response of E. huxleyi. 
Phosphorus scavenging 
Genes associated with phosphorus metabolism (Dyhrman et al. 2006; McKew et al. 2015) showed a global trend towards increased abundance following N addition. The –P condition had the most genes with significantly increased abundance within P-metabolism, with significant increases observed in genes associated with P transport, P scavenging from organic molecules, and polyphosphate (poly-P) metabolism. A family of vacuolar transport chaparonins (VTC), which are thought to be associated with poly-P metabolism (Ogawa et al. 2000; Hothorn et al. 2009; Dyhrman et al. 2012), had the largest significant fold change in each of the conditions to which N was added (Figure 4). Though thought to be a luxury uptake response (Perry et al. 1976), VTC has been observed to be to be increased in diatoms under P-limitation (Dyhrman et al. 2006, 2012) and may be indicative of internal poly-P cycling consistent with recent observation of bulk poly-P in the NPSG and the Sargasso Sea (Martin et al. 2014; Diaz et al. 2015). Genes associated with the scavenging of PO4 from organic molecules were also significantly increased, with two glycerophosphoryl diester phosphodiesterase (GDP) orthologus groups and a 5’ nucleotidase (NTD) orthologus group significantly more abundant following N addition (Figure 4). The protein of a 5’-nucleotidase was found to be present in E. huxleyi CCMP374 and CCMP373, and induced under P-limitation (Dyhrman and Palenik 2003). Additionally, the GDP detected here was found to be significantly (~20 fold) higher in a –P E. huxleyi strain CCMP1516 proteome than in replete or –N (McKew et al. 2015). This suggests that P-cycling from organic molecules such as nucleotides (both internal and external) may be central to its low P response. 5’-nucleotidase has also been observed to be significantly increased in diatom and palagophyte metabolism under P-limitation (Wurch et al. 2011b; Dyhrman et al. 2012), indicating that this metabolic strategy as seen across diverse taxonomic groups may be a deeply rooted response to P-limitation (Martiny et al. 2013).
It has been suggested that E. huxleyi may have one of the highest affinities (out of the eukaryotic algae) for P, leading to its success in P-limited competition experiments (Reigman 1992, 2000). Though initially less responsive to nutrient pulses than r-selected diatoms (Alexander, 2015, others) it has been hypothesized that their ability to grow well in P-limiting environments facilitates their blooming (Lessard 2005; note they say that it is both low N and low P). Bulk community alkaline phosphatase assays identified significantly increased (Tukey HSD, p < 0.05) AP activity in the –P condition, relative to +N, +DSW, and control no addition, suggesting the –P treatment was P-limited (Supplemental Figure Alkaline Phosphatase). Two gene families whose inductions under P-limitation are well-characterized in E. huxleyi, alkaline phosphatase (AP1) (Xu et al. 2006) and phosphate-repressible phosphate permease (PRPP) (Chung et al. 2003; Dyhrman and Palenik 2003; Dyhrman et al. 2006), were significantly increased only in –P (Figure 4). Alkaline phosphatase is a cell surface protein used for scavenging organic P from the environment and its induction in –P is seen in many diverse phytoplankton groups (Sakshaug et al. 1984; Dyhrman and Palenik 1997, 2003; Wurch et al. 2011a). AP1 in E. huxleyi has been shown to be increased ~1000-fold at the transcript-level when subjected to P-limiting conditions (Xu et al. 2006), and was found to constitute 3\% of all spectral counts in a P-limited proteomic data (McKew et al. 2015). Similarly PRPP has been found to be induced under –P conditions at both the transcript (Dyhrman et al. 2006) and protein (McKew et al. 2015) levels, as well as in cultures grown on organic nitrogen (Bruhn et al. 2010). These findings highlight the broad response to induced P-limiting conditions of E. huxleyi, in an oceanographic region that is not typically P-limited (Coleman and Chisholm 2010). This adaptability potentially underscores the importance of sexual reproduction in the success of E. huxleyi (Frada et al. 2008). 
There has been evidence to suggest that there may be a viral manipulation of host-PO4 uptake (Monier et al. 2012), as virus replication has high P requirements (Bratbak et al. 1993). This hypothesis is supported by the significant increase in the abundance of the orthologus group of putative dihydroceramide desaturase delta-4 (JGI \# 54601), which has close homologs in the genome of E. huxleyi virus 86.  A transcript within this orthologous group, belonging to the hyper-resistant strain CCMP379, was also found to be significantly increased by more than 9 log2 fold change in each of the conditions where N was added (Supplemental Table X).
Shift in life stage and calcification state
Tracking genes associated with ploidy state and calcification suggest that the community in the NPSG and in non-N-amended incubations were haploid and non-calcifyin. The addition of N, however, shifted the community to diploid and calcifying states (Figure 4). Genes thought to be associated with calcification (Mackinder et al. 2010) and found to be up-regulated in calcifying cells (Mackinder et al. 2011) were found to be significantly increased following N-addition. These genes included those associated with inorganic carbon transport (e.g. carbonic anhydrases (β, γ, δCA) and a group of anion (Cl / HCO3) exchangers (AE1)), calcium acquisition (e.g. voltage-gated Ca2+ channel (CAV), Na+/Ca2+-K+ exchanger (NCKX), and Ca2+/Mg2+-permeable cation channel (CX)), and proton transport (e.g. Vacuolar H+-ATPase V0 sector subunits c/c (ATPVcc)). Simultaneously in N amended incubations, there was a significant decrease in the abundance of two genes used as markers of haploid life phase (Frada et al. 2012), dynein heavy chain (DYH) and histone H2A (H2A). The life cycle of E. huxleyi is thought to fluctuate between diploid and haploid stages, with calcification only occurring in the diploid stage, and motility during the haploid stage (Paasche 2001). The coordinate increase of genes associated with calcification with the decrease of genes associated with haploid life stage, particularly DYH, which is integral to the flagella, highlight this known association of calcification and ploidy state. The pattern of increased calcification following N addition also falls in line with previously described coordination between nutrient environment and calcification in E. huxleyi (Paasche 2001). P-limitation has been observed to increase Ca content per coccolith and induce calcification in non-calcifying cultures, while N-limitation was found to decrease Ca content per cocolith (Paasche and Brubak 1994; Paasche 1998). The link between ploidy or life phase and nutrient concentration is not well understood (Green et al. 1996), though a connection to viral infection has been hypothesized (Frada et al. 2008). 
Conclusion
The broad distribution and success of E. huxleyi across many ecological gradients, points to a high degree of genetic and physiological variability within this species complex, such as is seen in other cosmopolitan phytoplankton such as Prochlorococcus or Ostreococcus (Derelle et al. 2006; Johnson et al. 2006). Microsatellite studies of field populations found high diversity within a population and low FST¬ between populations (Iglesias-Rodriguez et al. 2006), hinting that such global genetic variability might translate into variability in ecological or physiological function across various isolates. 
Metabolic plasticity in response to environmental change is a current cornerstone to the study of phytoplankton physiology, with much effort being put towards characterizing transcript- and protein-level shifts following perturbation (Dyhrman et al. 2006, 2012; Wurch et al. 2011a; Bertrand et al. 2012; Jones et al. 2013; Bender et al. 2014; Frischkorn et al. 2014). Moving the studies of metabolic plasticity into field has shed light on the molecular mechanisms underlying functional group differences (Alexander et al. 2015b), gradients of  co-limitation across ocean basins (Saito et al. 2011), and resource partitioning occurring between closely related species (Alexander et al. 2015a). An alternative (or compliment) to such metabolic plasticity may be genomic variability, which enables stable niche partitioning within cosmopolitan populations, such as has been observed in the unique transcriptional and physiological responses to stress amongst various Prochlorococcus ecotypes (Thompson et al. 2011).

Together, these data give us an unprecedented observation of the strain composition, strain-specific activity, and metabolic fingerprint of the community both in an oligotrophic system and under different nutrient regimes. 


Materials and Methods:
Sample collection and shipboard nutrient incubation experiments
Seawater for the in situ eukaryote community mRNA analysis was collected at the HOT, Station ALOHA (22˚45’ N, 158˚00’ W) from a depth of 25 m at 1400 hrs (local time) on six occasions during the summer of 2012 (S1: 6 August, S2: 12 August, S3: 24 August, S4: 30 August, S5: 2 September, S6: 5 September) using a Eulerian sampling scheme as part of the HOE-DYLAN research expedition as per Alexander et al. (2015b). Water was collected in acid-washed 20-L carboys and approximately 60 L of seawater was prescreened through 200 µm mesh and then filtered onto polycarbonate filters (5.0 µm pore size, 47 mm, Whatman) by way of peristaltic pump. Filters were changed every 20 minutes or when flow rate decreased. Filters were placed in cryovials and stored in liquid nitrogen until mRNA extraction. The total length of filtration time did not exceed 3 hours. 
In conjunction with these field-based surveys, two factorial nutrient amendment incubation experiments focused on the macronutrients N and P were performed with natural communities (T0 of E1 was T0 of E2 was S4) (STable 1). Incubations were modeled after a simulated 10\% deep seawater (DSW) upwelling as described in Alexander et al. (2015b) and designed to tease apart the potential nutritional components of DSW upwelling. The concentration of iron was modeled after Marchetti et al. (2012) and vitamin B12 was modeled after Bertrand (2007). Triplicate 20-L carboys of each treatment were incubated at 30\% surface light-levels using on-deck incubators for 7 days and processed as described above, on the final day at 1400 hrs (local time). Nutrient concentrations for phosphate [PO4], nitrate and nitrite [NO2+NO3] were measured by filtering 125 mL of seawater through a 0.2-μm, 47-mm polycarbonate filter, and stored frozen (−20 °C) in acid washed bottles until analysis at the Chesapeake Bay Lab at the University of Maryland according to the facility's protocols. Samples for alkaline phosphatase activity (APA) were collected by filtering 250-ml of whole seawater onto polycarbonate filters (0.2 µm pore size, 47 mm, Whatman) and frozen at -20oC. These filters were then resuspended in artificial seawater and assayed for APA fluorometrically using the fluorogenic phosphatase substrate 6,8-difluoro-4-methylumbelliferyl phosphate (diMUF-P, Molecular probes) following established field protocols (Dyhrman and Ruttenberg 2006). Chlorophyll a was measured on whole water samples collected onto GF/F filters (25 mm, Whatman) using a 90\% acetone extraction and assayed by fluorescence using the AquaFluor Turner TD700 (Parsons et al. 1984).
RNA Extraction and Sequencing
RNA was extracted from individual filters with the RNeasy Mini Kit (Qiagen), following a modified version of the yeast protocol. Briefly, lysis buffer and RNA-clean zirconia/silica beads was added to the filter and samples were vortexed for 1 minute, placed on ice for 30 seconds, and then vortexed again for 1 minute. Samples were then processed following the yeast protocol. The resulting RNA was eluted in water and then treated for possible DNA contamination using TURBO DNA-free Kit (Ambion) following the Rigorous DNase protocol. RNA from individual filters was then pooled by sample, using the RNA Cleanup Protocol from the RNeasy Mini Kit (Qiagen). The resulting RNA sample thus represented approximately 56 L of total seawater for the in situ sample. Filters were pooled across like triplicate bottles by treatment, totaling 56 L from each of the incubation treatments. The total RNA sample was then enriched for eukaryotic mRNA through a poly-A pull down. The resulting enriched mRNA sample then went through library preparation with the Illumina TruSeq mRNA Prep Kit (Illumina). Libraries were sequenced with the Illumina HiSeq2000 at Columbia Genome Center (New York, NY). Each sample was sequenced to produce a targeted 60 million, 100 base pair, paired end reads. Raw sequence data quality was visualized using FastQC and then cleaned and trimmed using Trimmomatic v 0.27 (paired end mode; 6-base pair wide sliding window for quality below 20; minimum length 25 base pair). 
Community and strain specific mapping and expression analysis
Transcriptome sequences and annotations generated through the Marine Microbial Eukaryote Transcriptome Sequencing Project (MMETSP) that were made public as of 17 March 2014 were collected and treated as per Alexander et al. (2015b) to track species composition of the metatranscriptomes. Due to the large size of the resulting MMETSP database, trimmed reads from the metatranscriptome were mapped to the MMETSP using the Burrows-Wheeler Aligner (Li and Durbin 2010) (BWA-mem, parameters: -k 10 -aM) and then counted using the HTSeq 0.6.1 package (Anders et al. 2014). 
The combined transcriptomes (as assembled from the NCGR on 4 September 2013) from unialgal cultures of Emiliania huxleyi strains CCMP374 (MMETSP1006-MMETSP1009), CCMP379 (MMETSP0994-MMETSP0997), CCMP370 (MMETSP1154-MMETSP1157), and PLYM219 (MMETSP1150-MMETSP1153). All transcriptome assemblies used are available through the iMicrobe data commons. Additionally, the predicted transcripts from the E. huxleyi genome, strain CCMP1516, were used. All transcriptomes were trimmed based on predicted peptide length, requiring sequences be longer than 70 amino acids. The resulting set of genes was considered for subsequent analyses. Peptide sequences were clustered into gene clusters with orthoMCL (citation), using standard parameters: BLASTP with an e-value cutoff of 1e-5, and an inflation value (-I) of 1.5. Initially, the transcripts unique to CCMP1516, here surveyed using the predicted transcripts from the genome, were the most dominant of the subsets of genes in these analyses, representing ~50\% of the E. huxleyi reads in the field (Supplemental Figure Core Changes). Closer inspection demonstrated that many of the most highly represented genes identified as unique to CCMP1516 were associated with metabolic stasis or senescence (e.g. OG1\_5\_1124, a group of homologous proteins in the E. huxleyi genome such as JGI \# 413698 annotated as putative senescence-related proteins and highly expressed in all field samples). Many of the proteins in the unique set of CCMP1516 were identified as “core” amongst the 13 strains surveyed by Read et al. (2013), yet were absent in some or all of the transcriptomes of the four strains in this study. This absence likely is related to the fact that these strains were largely sampled under exponential growth conditions, limiting the expression of genes that might be associated with stressors or stasis. The lack of ‘core’ gene representation in some of these transcriptomes underscores the importance of growth condition in transcriptome complete
Using this clustering framework, field and incubation samples were mapped to the data set using RSEM, a software package designed to estimate gene and isoform expression values from RNA-seq data. Here we define orthologous groups as genes and individual transcripts (from any strain) as isoforms. Data were mapped using RSEM version 1.2.20 (parameters: --paired-end –p8 –bowtie2 –bowtie2-mismatch-rate 0.2). A note: RSEM is not yet able to deal with gapped mapping, such as enabled by bwa, which was used for the community-level mapping due to database size constraints. Taking a conservative approach, the RNA abundances from like treatments (each consisting of pooled triplicate bottles), which were run with different communities from separate water masses more than two weeks apart, were considered to be biological replicates for differential abundance analysis. These analyses were run with ll using default parameters to calculate dispersion and assess differential abundance of both individual transcripts and orthologous groups of each of the amended incubations compared to the no-addition control. 
Figure Legends: 
Figure 1. Population shifts in response to nitrogen addition for both haptophytes and diatoms differ with addition of trace metals. The relative **********
Contribution of the most abundant diatom and haptophyte genus to all mapped reads across six in situ samples and two incubation experiments. The percentage of all mapped reads corresponding to the most dominant diatom (grey line, squares) and haptophyte (black line, circles) genus in each of the in situ samples (A) and in each of the two replicated incubation experiments, E1 (B) and E2 (C). The taxonomic designation of the most abundant taxa is indicated by the color of the shape. Nutrients added to incubation experiments are indicated on the exterior of the radar plots, indicating the addition of nitrate, phosphate, trace metals including FeCl, and vitamins.
Figure 2. The distribution, orthologus grouping, and relative representation of E. huxleyi strains in the field and in incubation experiments. The isolation location of the five E. huxleyi strains used in this study (A). Genes were clustered based on protein homology using OrthoMCL to identify the “core” orthologus groups (OGs) that were common to all strains, OGs that were present in some but not all strains (Mix), and OGs unique to each of the strains (B). The relative contribution of the unique gene sets were tracked across time in the field (C) and in each of the incubation experiments, E1 (D) and E2 (E). Nutrients added to incubation experiments are indicated on the exterior of the radar plots, indicating the addition of nitrate, phosphate, trace metals, and vitamins.
Figure 3. Principle components analysis of the RSEM estimated strain-specific contributions to the observed abundance of each of the 5243 orthologus group common amongst the five strains in the field. A PCA was performed on the RSEM estimated TPM abundance of strain-specific transcripts within the 5243 shared orthologous groups across all in situ and incubation samples. The first three components (with a combined 36.9\% of variance explained) are plotted against each other, Component 1 vs Component 2 (A, C) and Component 2 vs Component 3 (B, D). Plots are colored both by strain (A, B) and by condition (C, D). 
Figure 4. Fold change of genes associated with nitrogen and phosphorus metabolism, calcification, and ploidy across each of the incubation amendments compared to the no addition control. The log fold change of orthologus groups associated with N and P metabolism, calcification, and ploidy state was assessed with edgeR across the five ammended incubations compared to the no addition control are plotted in opaque grey. The size of the orthologus group marker is proportionate to the log of the mean abundance across the two treatments. Orthologus groups are that are significantly differentially abundant (FDR < 0.05) are plotted highlighted in red. Individual transcripts within an orthologous group are plotted in light grey or red to indicate significance of fold change. Genes of interest are labeled with abbreviations as follows, labels in bold indicate significant regulation in two or more conditions. Acetamidase, formaidase, indole acetamide hydrolase (Amidases); Acety-CoA carboxylase (ACCase); Ferredoxin-dependent glutamate synthase (GLT); Nitrate transporter (NRT); Urease (URE); Urea transporter (UTP); Glutamine synthase type II (GS type II); Glutamine synthase type III (GS type III); Ammonium transporters (AMTs); Foramte/nitrite transporter (NAR); Glycerophosphoryl diester phosphodiesterase (GDP); Vacuolar transport chaperone (VTC); 5’-nucleotidase (NTD); Alkalin phosphatase (AP1); Phosphate repressible phosphate permease (PRPP); Alternative oxidase (AOX); Plasma membrane H+ATPase (ATPase); Vacuolar H+-ATPase V1 sector, subunit B (ATPVB); Ca2+/Mg2+-permeable cation channel (CX); Glutamic acid, proline, and alanine rich Ca2+ binding protein (GPA); Anion (Na+-independent Cl / HCO3) exchanger (AE1); β-type carbonic anhydrase (βCA), δ-type carbonic anhydrase (δCA), γ-type carbonic anhydrase (γCA); voltage-gated Ca2+ channel (CAV); Na+/Ca2+-K+ exchanger (NCKX); Vacuolar H+-ATPase V0 sector subunits c/c (ATPVcc); Dynein heavy chain (DYH); Histone H2A (H2A). 
Supplemental Figure Legends: 
SupplementalFigure\_AnnotationOfCommonGenes\_v1. Annotation of orthologous groups using KOG orthology for all E. huxleyi orthologous groups and for shared orthologous groups. The relative percentage of orthologous groups able to be annotated for all orthologous groups (A) and orthologous group shared amongst the five studied strains (B) are shown. The number of orthologous groups falling into each KOG class or multiple classes (m.c.) is shown for both all orthologous groups (C) and shared groups (D). 
SupplementalFigure\_APAperChlorophyll. Bulk community alkaline phosphatase activity normalized to chlorophyll at the time of RNA sampling (7 days) for each of the six conditions in E2. Alkaline phosphatase activity (APA) for the community > 0.2 μm was measured with a fluorimetric approach, using a soluble DOP substrate analogue (DiFMU).  APA was normalized to total chlorophyll for each of the six treatments across triplicate bottles (n=3). Significant differences are shown by Tukey HSD, with a and b indicating significant differences at 0.05 confidence. 
SupplementalFigure\_core\_changes\_v1. The relative expression of ‘core’, shared, and CCMP1516-specific transcripts across time and in incubation experiments. The percentage of all mapped reads corresponding to the genes considered to be ‘core’ by Read et al. (2013) (black), found to be shared across the five strains used in this study (grey), or originally considered to be unique to CCMP1516 are plotted for each in situ sample (A) and in each of the two replicated incubation experiments, E1 (B) and E2 (C). Nutrients added to incubation experiments are indicated on the exterior of the radar plots, indicating the addition of nitrate, phosphate, trace metals, and vitamins.
SupplementalFigure\_GeneDistrubtion. The number of orthologous groups falling into each of the possible strain sets. The relative strain membership is depicted in a scatter plot along the x-axis ranging from the first row of ‘shared’ or ‘core’ genes, common to all strains (black), through variable memberships across some but not all strains, to sets comprised of only one strain (colored). Genes common to all strains in this study are shown in black. Genes identified as ‘core’ in CCMP1516, the genome strain, by Read et al. (2013), but that were not identified in some or all of the other strains were added to the ‘shared’ set and are indicated in yellow hatching. 
SupplementalFigure\_KOG\_Distribution\_AcrossTranscriptomes. The percent of genes falling into each of the KOG classes for each of the strains in this study.  
SupplementalFigure\_MANTA\_Gene\_Isoform. Log normalized fold change plotted against log normalized average abundance for each of the five amended treatments compared to the no-addition control. edgeR was used to assess the average abundance and log fold change for each of the orthologous groups (left column) and strain-specific transcripts (right column). Genes are colored by generalized metabolic function. The intensity of the color indicates significance. 
SupplementalFigure\_nutrients\_v1. Inorganic nitrogen and phosphorus concentrations at the final time (7 days) for each of the six treatments in E1 and E2, averaged across triplicate bottles (n=3) . 
SupplementalFigure\_SharedGeneComp. The RSEM estimated contribution of each strain to the abundance of the shared set of genes in the field and incubation experiments. 
References:
Alcolombri, U., S. Ben-Dor, E. Feldmesser, Y. Levin, D. S. Tawfik, and a. Vardi. 2015. Identification of the algal dimethyl sulfide-releasing enzyme: A missing link in the marine sulfur cycle. Science (80-. ). 348: 1466–1469.
Alexander, H., B. D. Jenkins, T. A. Rynearson, and S. T. Dyhrman. 2015a. Metatranscriptome analyses indicate resource partitioning between diatoms in the field. Proc. Natl. Acad. Sci. U. S. A. 112: E2182–E2190.
Alexander, H., M. Rouco, S. T. Haley, S. T. Wilson, D. M. Karl, and S. T. Dyhrman. 2015b. Functional group-specific traits drive phytoplankton dynamics in the oligotrophic ocean. Proc. Natl. Acad. Sci. 112: 201518165.
POC ratio in the coccolithophore Emiliania huxleyi grown under light-limiting conditions and different daylengths. J. Exp. Mar. Biol.

\section{Materials and Methods}
\section{Results and Discussion}
