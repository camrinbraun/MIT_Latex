
\chapter{Introduction}
\raggedbottom
\clearpage


%\section{Background}
%initial hypotheses about where animals go and driving question(s)
Marine ecology, particularly of the open ocean, is the study of marine organisms in the context of each other and the environment. The open ocean is characterized by a range of temporal and spatial scales that are manifested as a complex integration of physical, chemical and biological phenoma. Ocean physics and biology interact to generate the behaviors we observe, yet animal movements are typically observed at basin scales, revealing fascinating long-distance migrations, or at very small scales when, for example, they interact with fishing effort. Small and intermediate scales are poorly understood, however perhaps the most important biophysical interactions occur at and below the oceanic (sub)mesoscale with spatial scales of $O(1 - 100\ km)$. This lack of knowledge stems from issues inherent in studying animal movements in an opaque medium (such as seawater), particularly with constraints on geolocation of individual fish as they move around. 

Since the first tag-recapture experiments on Atlantic salmon in 1873 \citep{Everhart1975}, scientists have been actively studying animal movements. Where animals go and what they do there is of fundamental importance for everything from ocean properties, such as carbon export \citep[\eg][]{Lavery2010a}, to ecosystem structure \citep[\eg][]{Thorrold2014} and fishery dynamics \citep[\eg][]{Block2005}. Yet, despite significant advances in tag technology, studying animals that move thousands of kilometers while diving from the surface to bathypelagic depths (> 2000 m) all while immersed in an opaque medium like seawater remains a formidable challenge. Our current understanding of large pelagic fish behavior is largely limited to large-scale movements and behavior summaries derived from light-level geolocation by archival tags. The accuracy of light geolocation ($\pm$100-200 km; ~10,000 km\textsuperscript{2}) and its depth limits (< 100 m) constrain its application to large-scale surface movements. Yet, many fishes exhibit diving behavior that renders light geolocation impossible, and the magnitude of error precludes our ability to resolve important ecological dynamics such as habitat association (\eg specific bathymetric or oceanographic feature use). Therefore, improved analytical techniques and different approaches to tracking studies are necessary to acquire more accurate position information with which we can characterize the ocean environment and provide context for the animal behaviors we observe. Thus, I focus on the following questions in this thesis:

%enumerate questions
\begin{enumerate}
    \item How can we improve existing methods of geolocation within the constraints of existing archival tag technology?
    \item Can we use improved methods to leverage historical data for use in modern physical-biological interaction studies, and, by doing so, improve the inference we can gain from this data?
    \item How can we apply the lessons learned from the first two questions to design a study that alleviates the limitations inherent in these approaches such that we can quantify use of specific oceanographic features?
\end{enumerate}

\section{It's a big ocean and approaches to locating animals}
% animal movement in general and why its historically been difficult to follow things in the ocean
The study of marine animal movements remains a formidable challenge due to the difficulty of making observations on elusive species in a large ocean. Information on large pelagic fishes in the open ocean is particularly lacking, despite lucrative fisheries and intense fishing effort for a number of these species. Basin-scale movements \citep{Skomal2009} and deep forays to meso- and bathypelagic depths \citep{Thorrold2014a} further complicate this research by making individuals even less available for direct observations than coastal fish species.

Historically, insights into pelagic fish ecology were limited to those obtained visually at the surface. These constraints allowed access to only a small portion of any species' lifetime movements and behavior, and it has long been recognized that novel approaches are needed to better understand the ecology of these fishes. Mark-recapture techniques employing external tags have been used extensively since the early twentieth century, but this approach only provides deployment and retrieval locations with no data on a fish's behavior in the interim \citep{Kohler2001}.

The development of electronic tags in the mid-1950s has since enabled researchers to gather a more holistic view of the behaviors of large pelagic organisms. Temperature and pressure sensors developed simultaneously in the 1970s and transformed our ability to follow the incredible dives that many species perform \citep{Carey1981}. Satellite transmitting tags followed in the 80s and allowed accurate positioning of surface-oriented species such as basking sharks \citep{Priede1984}. Data storage, or archival, tags were first deployed in the 1990s and have since been widely used worldwide \citep{Hussey2015}. Modern tags incorporate multiple sensors (including acoustics, accelerometers and cameras) and can release themselves from a study animal, effectively eliminating the fishery-dependent nature of earlier studies. While other tracking technologies exist, the two most widely used technologies in aquatic ecosystems today are acoustic and satellite-based tags \citep{Hussey2015}.

Acoustic techniques proliferated starting in the 1960s and have since made many significant contributions to the field \citep[\eg][]{Carey1981, Carey1990}. These techniques rely on transmission of acoustic signals by tagged animals that are logged at moored or mobile receiving stations. Acoustic studies now boast tags that last up to 10 years, typically exhibit error on the order of meters and, in some cases, can leverage satellite-linked loggers that eliminate the need to recover devices to access stored data \citep{Donaldson2014}. The benefits of acoustic telemetry have driven exceptional growth in acoustic monitoring studies \citep{Hussey2015}; however, current acoustic studies are limited to < 2 km range from receiving stations (\eg hydrophones) which proves intractable for studying highly migratory species in the open ocean. Thus, the major limitation with this approach is scale which is perhaps among the most interesting ocean challenges \citep{Stommel1963, Haury1978}. 

In the late 1980s, satellite-linked archival tags followed the advent of acoustic devices. This technology alleviated the \is infrastructure, and thus scaling, issues associated with acoustic telemetry, enabling researchers to follow fish that move thousands of kilometers across ocean basins in a matter of months, such as tunas \citep{Block2005} and pelagic fishes and sharks \citep[\eg][]{Block2011}. Since their inception, these tags have become increasingly relevant \citep{Hussey2015} for studying horizontal and vertical movements \citep{Block2011, Thorrold2014a, Berumen2014}, residency \citep{Domeier2006}, mortality \citep{Musyl2011a}, aggregative and feeding behaviors \citep{Jorgensen2012}, and other aspects of the biology and ecology of marine organisms. Pop-up satellite archival transmitting (PSAT) tags, specifically, have been used with great success on a number of taxa \citep{Hussey2015}. These devices are attached externally to a study animal and are programmed to pop-up from the animal after a predetermined deployment period. While active on the animal, these tags typically collect an \is time series of depth, temperature and light levels. Miniaturization and ever-improving sensors, batteries and storage capabilities make this a rapidly advancing field that has already demonstrated vast improvements over previous techniques. As such, this technology has been deployed on thousands of study animals encompassing nearly all marine taxa large enough to carry a tag \citep{Hussey2015}. These studies have now consequently described the broad-scale horizontal and vertical movements for many marine species. However, fewer studies attempt to perform quantitative analyis of how species interact with and choose to occupy the surrounding oceanographic environment \citep[except see, for example, ][]{Lawson2010}.

\section{I tagged some fish, now what?}\label{sec:geo}
% analytical approaches to tag data and geolocation
Despite all these advances, we remain limited by the fundamental physics of transmitting electromagnetic radiation (the way we communicate with satellites such as GPS) through an opaque medium such as seawater. As such, even cutting edge satellite technologies for fish still archive light data for geolocation. Ambient light records are used to estimate dawn and dusk from which longitude (local noon) and latitude (local day length) are derived to estimate positions \citep{Hill1994, Hill2001}. Early on, the community recognized the need to supplement light-based geolocation with other information, and we have since made sizable advances in analytic approaches to improved geolocation despite the age-old constraints. 

% Here's a synopsis of what people have done...hill, sibert, lam, nielsen, pedersen, thygesen, etc
Arguably the most important advances in the analysis of animal movement data, including the geolocation problem, has been state-space models which estimate the "state" of an unobserved process from an observed dataset \citep{Jonsen2013}. 

Perhaps the most notable early advance was the Kalman filter (that has applications from natural sciences to economics) by \cite{Sibert2001} being used to analyze position estimates from noisy light data \citep{Hill2001}. Further advances were quickly achieved by incorporating a comparison of \is SST recorded by the tag and remotely-sensed measures \citep{Teo2004, Nielsen2006}. Improvements since have expanded rapidly to incorporate other ancillary data such as tidal variation \citep[\eg][]{Metcalfe1997} to inform geolocation and more sophisticated modelling techniques such as hidden Markov models \citep[\eg][]{Pedersen2008}. The integration of \is measurements onboard archival tags with ever-improving statistical techniques and modern tools such as high-resolution oceanographic models and a suite of remote-sensing satellites promises to yield significant improvements in estimating animal movements over geolocation with light levels alone.

\section{Why did they do that?}
%Biophysical interactions in the ocean}
% biophysical interactions in the ocean & drivers of animal movements


% start with general dynamics and structuring of ocean ecosystems
Of particular relevance to this thesis is the role that oceanography plays in the structuring of pelagic ecosystems. Fronts and mesoscale eddies are among the most important features in the open ocean \citep{Chelton2011, McGillicuddy2016, Mahadevan2016}, and recent advances in satellite oceanography have allowed the automated identification and tracking of these features globally \citep{Chelton2011, Belkin2009}. Advances in our ability to observe and track these features have revealed rich regional variability in how these features influence lower trophic levels \citep{McGillicuddy2016, Gaube2017DSR} and have shown the potential coupling of biology and ocean physics that can lead to the formation of biological hotspots \citep{Mann2006, Belkin2014}.

%and have most recently led to studies of how these features might influence animal populations \citep[\eg][]{Kobayashi2011, Gaube2017, Belkin2014, Queiroz2016}.

% very brief what we know about non-fish
Electronic tag technologies permit quantitative analyses of the use of these features by pelagic predators. Due to geolocation constraints (see Section \ref{sec:geo}), the vast majority of advances have been made on obligate surface-oriented taxa such as turtles \citep[\eg][]{Gaube2017, Polovina2006, Kobayashi2011}, marine mammals \citep[\eg][]{Johnston2007, Bailleul2010} and birds \citep[\eg][]{Thorne2013, TewKai2009}. Together, these studies suggest the most important features for pelagic predators are likely to be associated with enhanced vertical flux of nutrients leading to increases in primary production \citep{Franks1992}. Convergent flow at front boundaries and along the periphery of eddies can also aggregate passive particles, including phytoplankton, and these areas are thought to attract pelagic predators due to increased foraging opportunities, generating hotspots \citep{Scales2014}.

% history of biophysical interactions for fishes, focus on HMS. demonstrate how this is traditionally done w/ catch data which is inherently flawed. incorporate, for example, hobday & hartog

Historically, anecdotal evidence and fisheries statistics have supported the association of pelagic fishes with physical structures like fronts and eddies \citep[e.g.,][]{Hobday2014}, but scientists currently understand very little about the biology of these important oceanographic features, particularly for fish communities. Technological limitations inherent in light-level geolocation \citep{Braun2015} have largely precluded robust analyses of the associations between (sub)mesoscale oceanographic structures and pelagic fishes. Conventional light-level geolocation from analysis of archival tag data is generally insufficiently accurate to match fish movements with specific mesoscale features detected from satellite observations. As a result, we understand little about the biophysical factors influencing the ecology of large pelagic fishes. Despite these constraints, some progress has been made, primarily using fisheries data. For example, \cite{Hsu2015} compared catch data from the U.S. northwest Atlantic longline fishery to the mesoscale eddy field in this region and found bluefin tuna associated with anticyclonic mesoscale eddies while swordfish were more often outside of eddies. Similarly, using catch data, \cite{Hobday2014} found southern bluefin tuna associated with anticyclones off Australia while opah preferred cyclonic eddies. Similar analyses have been conducted using other fisheries, species and ocean features, such as fronts \citep[\eg][]{Worm2005}.

A few recent studies have used tracking data to investigate overlap between pelagic fishes and ocean features. For example, \cite{Miller2015} showed basking sharks prefer productive regions of the Northeast Atlantic characterized by contemporaneous thermal and chlorophyll fronts. And \cite{Gaube2018} showed two tagged white sharks associated with anticyclonic eddies of the Northwest Atlantic, suggesting these features may influence foraging opportunities. However, analyzing fisheries-independent tracking data in the context of (sub)mesoscale oceanography remains an anomalous approach due largely to geolocation issues mentioned above. In addition, few studies of fish movements have been specifically designed to investigate feature use. Thus, linking movements of pelagic fishes with (sub)mesoscale physical-biological mechanisms remains largely unstudied but is critical to understanding the structuring of pelagic ecosystems and to designing appropriate management approaches for pelagic fish populations.

\section{Why should we care?}
%Practical considerations and applications}
 % primarily to fisheries management problems
 
Over the past century, human exploitation of natural systems has propagated throughout the ocean, subjecting large marine predators to intense exploitation \citep{Byrne2017} and eliciting disproportionate effects on large vertebrates \citep{Jackson2001, Baum2003}. As a result of intense anthropogenic impact, many predatory fish populations (including billfish, tuna and sharks) have joined the ranks of immediate conservation concern, and their populations are routinely depleted by 50-70\% \citep{Hilborn2003} and up to 90\% \citep{Myers2005}. Many of these species spend most of their life in the open ocean and traverse vast expanses of water in search of food, reproductive opportunities and suitable habitat. Their behavioral tendencies subject them to fishing pressure by an array of different gear types and exploitation levels under different national jurisdictions and throughout the high seas. In addition, the current lack of information about key life history traits, population size, movements and habitat use of these fishes amplifies the problem of managing these populations as anthropogenic pressures on fishes continue to rise \citep{Dulvy2008, Ferretti2010}.

A central challenge to the management of ocean seascapes is the dynamic spatial and temporal nature of ocean systems \citep{Lewison2015}. Yet, traditional ocean management approaches are most often static. Static management approaches may be less effective for managing highly migratory species and are less able to respond to changing ocean dynamics at scales from ephemeral features, such as eddies, to chronic climate-induced changes in our ocean. Relatively simple time series of animal movements and behavior coupled with modeled and/or remotely-sensed representations of the marine environment, as described in this thesis, can be combined to develop tools for near real-time prediction of species-specific habitat use. These dynamic approaches will be critical for managing fisheries in a changing ocean \citep{Maxwell2015}, but necessitate some understanding of physical-biological mechanisms governing the observed animal behaviors.

Filling these gaps will be essential for formulating effective management plans \citep{Cullis-Suzuki2010}, understanding the potential effects of climate change \citep{Hazen2012} and ensuring continued harvest of these resources \citep{Pauly1998, Watson2013}. Additionally, an improved understanding of the behaviors of large pelagic fishes will not only inform us about the ecology of the taxa themselves, but may also facilitate broader understanding of biogeochemical processes in the ocean \citep{Lavery2010a, Roman2010}.

\section{Thesis Overview}
% overarching goal
%This thesis quantifies the movement and oceanographic association of pelagic fishes.

% chapter-specific list-like paragraph
with the aforementioned questions in mind, this thesis seeks to determine the oceanographic associations of large pelagic fishes in the North Atlantic Ocean in order to better understand species ecology and ecosystem dynamics and to inform improved fisheries management efforts. In chapter \ref{chap:2}, I developed significant methodological improvements for geolocating fishes equipped with traditional archival tags. Blue and mako sharks were instrumented both with PSATs and an independent Doppler-based satellite tag from which "known" locations were used to quantify error in the resulting PSAT geolocation model. Leveraging three-dimensional data from tags in conjunction with high-resolution oceanographic models facilitated significant improvements in error estimates relative to existing model frameworks. In chapters \ref{chap:3} and \ref{chap:4}, I applied this modeling technique to basking shark and swordfish datasets, respectively. These two species have proven particularly difficult to track due to significant occupation of the aphotic regions of the ocean resulting in little to no light data available for geolocation. In \cref{chap:3} I used the more accurate locations to quantify large-scale movements, seasonality and vertical habitat use of > 50 basking sharks in the western Atlantic. Improved swordfish tracks in \cref{chap:4} were also used to investigate movements and seasonality and, in some cases, were accurate enough to describe mesoscale feature use. I also mined fisheries-dependent data for swordfish in the North Atlantic in \cref{chap:4} that I synthesized with fishery-independent electronic data to quantify oceanographic influences on swordfish. Finally, \cref{chap:5} focused on building a robust dataset with which I could test interactions between pelagic predators and mesoscale eddies. I double-tagged 15 blue sharks (as above) in order to reconstruct 3-D movements in the Gulf Stream eddy field and collocated these data to remotely-sensed and modeled oceanographic data to quantify shark-eddy interactions. Overall, I demonstrate that integrating observations of animal movement and behavior with satellite imagery and physical data provides significantly enhanced insight on habitat preferences and the interactions of an animal with its environment.

%--------------------------------
\begin{comment}


drawing conclusions about the function of diving (block electronic tagging book) and oceanographic associations since the first quality datasets from electronic tags in the 1990s

bakun around p. 6 for discussion of "scales" and pattern

% ** see good reviews such as hobday and hartog or mcgillicuddy ann rev mar sci **

As a result, we understand little about the biophysical factors influencing the ecology of large pelagic fishes.


%Mesoscale eddies and submesoscale fronts make up the internal weather of the ocean, exciting vertical fluxes and transporting pelagic communities hundreds to thousands of kilometers. Yet, while the application of satellite technology now allows us to view these features in almost real time, the influence of these structures on pelagic predators remains largely unknown. With the latest satellite tagging technologies, we now have the ability to observe the movement of large oceanic predators in high-resolution, three-dimensional space. Here, we propose to investigate the use of mesoscale eddies and meanders and submesoscale fronts by pelagic predators in the North Atlantic by collocating trajectories obtained from satellite-tagged sharks with (sub)mesoscale structures, defined here as features with spatial scales of $O(1 - 100\ km)$, identified and tracked in maps of sea level anomalies and sea surface temperature. The collocation of individual sharks, as model apex predators, with (sub)mesoscale ocean features will allow us to understand how these predators use these ubiquitous structures. Furthermore, by comparing observed patterns of feature use by the predators to satellite observations of ocean currents, surface temperature, and ocean color, I link observed behavior to known (sub)mesoscale physical/biological processes. Thus, the goal of the research proposed here is to determine the influence of (sub)mesoscale oceanographic features on the movements of pelagic predators, allowing us to link predator behavior to (sub)mesoscale physical/biological mechanisms, through the synergistic analysis of individual fish movement and concurrent satellite observations of (sub)mesoscale eddies, meanders and fronts.

% Of particular interest to this study is the role that (sub)mesoscale oceanography plays in the structuring of pelagic ecosystems and the movement ecology of large pelagic fishes. Historically, anecdotal evidence and fisheries statistics have supported the association of pelagic fishes with mesoscale structures like fronts and eddies \citep[e.g.,][]{hobday2014derived}, but scientists currently understand very little about the biology of these important oceanographic features. Furthermore, little data is available on fish communities that inhabit these structures. Electronic tag technologies now permit quantitative, fisheries-independent analyses of the use of these features by pelagic predators. Recent work has shown that obligate surface-dwelling marine organisms (e.g. turtles, seals, and birds) associate with persistent mesoscale fronts (reviewed in \cite{scales2014review}), and limited research on fishes has shown the importance of seasonally persistent fronts for basking sharks \citep{miller2015}. Technological limitations inherent in light-level geolocation \citep{braun2015movements} have, however, largely precluded robust analyses of the associations between (sub)mesoscale oceanographic structures and pelagic fishes. Conventional light-level geolocation from analysis of archival tag data is generally insufficiently accurate to match fish movements with specific mesoscale features detected from satellite observations. However, fin-mounted tags capable of generating positions by interrogating ARGOS satellites provide sufficiently accurate positions for (sub)mesoscale collocation analyses. In addition, this technology can be paired with conventional PSAT tags that provide detailed information on the vertical movements and water column properties. Together, the tags can provide high-resolution $(\sim 60\ sec)$ measurements of vertical movements with a spatial accuracy appropriate for matching resulting tracks to concurrent (sub)mesoscale oceanographic structures.




which involved integrating SST...that achieved so much ...numbers...improvement over light alone.



However, by integrating additional \is measurements onboard archival tags ...

can now get geolocation down below theoretical light thresholds using modern tools such as high-resolution oceanographic models and a suite of remote-sensing satellites. in combination, we're now better able than ever before to contextualize the observations we make using tagged animals that can teach us both about the animals and about the ocean.


The issue of scale is described by \cite{Haury1978} as the "great variability...on all scales in space and time caused by patchiness." This refers to the nested nature of ocean phenomena and the complexity in interpreting an observation in...by itself...


that are intimately tied to 

Physical-biological interactions that drive the behaviors we observe.

physical mechanisms

Processes observed in the open ocean are an integration of physical, chemical and biological phenomena that

The ocean is a complex medium, characterized by a range of scales
the issue of scale in the ocean

AND

animal movements at basin-scale and long time periods, relatively well understood and revealed fascinating migrations. enter history stuff like original basker paper and carey's work

BUT

scales below hundreds/thousands of kms and months are poorly understood. yet, this is where the interesting stuff happens. this stems from issues with opaque medium and geolocation.

THEREFORE

due to tech and methodological improvements, i can probably do better. In this thesis, i seek to:

Improvements in satellite tag technology and advances in analytical techniques would facilitate acquisition of better position information with which we could contextualize animal behaviors in their environment. 





\subsection{Aquatic telemetry}
% progress to date but what remains to be done. bit of history is good here

% animal movement in general and why its historically been difficult to follow things in the ocean
The study of animal movements in the ocean remains a formidable challenge due to the difficulty of making observations in an opaque medium, only a fraction of which can we directly observe. We understand relatively little about the ecology of marine fishes when compared to terrestrial taxa. Information on large pelagic fishes that occur in the open ocean is particularly lacking, despite lucrative fisheries and intense fishing effort for a number of these species. Basin-scale movements \citep{Skomal2009} and deep forays into the meso- and bathypelagic layers of the open ocean \citep{Thorrold2014a} further complicate this research by making individuals even less available for direct observations than coastal fish species. As a result, we understand little about the biophysical factors influencing the ecology of large pelagic fishes.

% aquatic telemetry, focus on sat telemetry
Historically, insights into pelagic fish ecology were limited to those obtained visually and at the surface. These constraints allowed access to only a small portion of any species' lifetime movements and behavior, and it has long been recognized that novel approaches are needed to better understand the ecology of these fishes. Mark-recapture techniques employing external tags have been used extensively sincet he early twentieth century, but this approach only provides deployment and retrieval locations iwth no data on a fish's behavior in the interim \citep{Kohler2001}.

The development of electronic tags has enabled researchers to gather a more holistic view of the behaviors of large pelagic organisms. Acoustic techniques proliferated starting in the 1960s and have since made many significant contributions to the field \citep[\eg][]{Carey1981}. Satellite-linked archival tags followed in the late 1980s and have enabled further important advances. Since their inception, these tags have become increasingly relevant \citep{Hussey2015} for studying horizontal and vertical movements \citep{Block2011, Thorrold2014a, Berumen2014}, residency \citep{Domeier2006}, mortality \citep{Musyl2011a}, aggregative and feeding behaviors \citep{Jorgensen2012}, and other aspects of the biology and ecology of marine organisms.

Pop-up satellite archival transmitting (PSAT) tags, specifically, have been used with great success on a number of taxa \citep{Hussey2015}. These devices are attached externally to a study animal and are programmed to pop-up from the animal after a predetermined deployment period. While active on the animal, these tags typically collect an \is time series of depth, temperature and light levels. Miniaturization and ever-improving sensors, batteries and storage capabilities make this a rapidly advancing field that has already demonstrated vast improvements over previous techniques. As such, this technology has been deployed on thousands of study animals encompassing nearly all marine taxa large enough to carry a tag \citep{Hussey2015}. These studies have now consequently described the broad-scale horizontal and vertical movements for many marine species. However, fewer studies attempt to perform quantitative analyis of how species interact with and choose to occupy the surrounding oceanographic environment \citep[except see, for example, ][]{Lawson2010}.


\subsection{Pelagic fish ecology}
% importance of these species. why care about HMS? finish with large-scale looming question
% impressive movements (horiz and vert). commercially relevant.


\subsection{Practical considerations}
Over the past century, human exploitation of natural systems has propagated throughout the ocean, subjecting large marine predators to intense exploitation \citep{Byrne2017} and eliciting disproportionate effects on large vertebrates \citep{Jackson2001, Baum2003}. As a result of intense anthropogenic impact, many predatory fish populations, including billfish, tuna and sharks, have joined the ranks of immediate conservation concern, and their populations are routinely depleted by 50-70\% \citep{Hilborn2003} and up to 90\% \citep{Myers2005}. Many of these species spend most of their life in the open ocean and traverse vast expanses of water in search of food, reproductive opportunities and suitable habitat. Their behavioral tendencies subject them to fishing pressure by an array of different gear types and exploitation levels under different national jurisdictions and throughout the high seas. In addition, the current lack of information about key life history traits, population size, movements and habitat use of these fishes amplifies the problem of managing these populations as anthropogenic pressures on fishes continue to rise \citep{Dulvy2008, Ferretti2010}.

dynamic ocean management

Relatively simple time series of continuous geolocation data and information on individual diving and ambient temperature from satellite tags can facilitate robust analyses and interpretation of individual movements. These data, in combinatoin with auxiliary enviornmental data and advanced analysis techniques, provide a unique opportunity to answer many unresolved questions regarding behavior and ecology of large pelagic fishes. Filling these gaps will be essential for formulating effective management plans \citep{Cullis-Suzuki2010}, understanding the potential effects of climate change \citep{Hazen2012} and ensuring continued harvest of these resources \citep{pauly1998, Watson2013}. Additionally, an improved understanding of the behaviors of large pelagic fishes will not only inform us about the ecology of the taxa themselves, but may also facilitate broader understanding of biogeochemical processes in the ocean \citep{Lavery2010, Roman2010}.

Part of the paucity of information governing our understanding of deep-water occupation by large pelagic fishes is due to the analytical limitations associated with PSAT tag data. While they provide high-resolution information on vertical behaviors, accurate horizontal positions require frequent surface occupation. This presents a significant constraint on studying marine species, many of which leave the surface for months at a time \citep{Skomal2009}.




% what this technique has taught us about movements of fish.
    % while impressive movements (horiz and vert), lot of work to be done to, for example, undertsand mechanisms driving movements.

% the single most important barrier to this has been error in fihs geolocation. here's why this is an issue.

% brings us to the questions? 
    % 1) how to better geolocate animals using oceanographic data? 
    % 2) oceanographic influences on animal movements? biophysical interactions?
    % 3) dynamic oceanography drive biology, the formation and degradation of hotspots, management?

    % ?? 2) how can improved geolocation techniques inform species ecology? fisheries oceanography? biophysical interactions



\end{comment}


%========================
%% END
