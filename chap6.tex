\chapter{Conclusion and Outlook}
\label{chap:6}
\clearpage
\raggedbottom

%\section{Thesis summary and next steps}
% in general, this section should provide some brief synthesis of your work before providing comments on limitations of your work, future directions, questions that arise from your results, etc

%% set the stage: broadly describe in P1 the context within which your thesis operates


%% what does your thesis do? typically you provide ~1 sentence summary of each chapter within a slightly broader narrative of "what it all means". this may be a few paragraphs.


%% in summary, what have we learned from this thesis? what challenges/questions remain?
\section{Summary}

Historically, observing movements of marine fishes has been limited by error in geolocation. These complications have rendered contextualizing observed behaviors and subsequent inference difficult. Consequently, oceanographic associations of many commercially-important or conservation-relevant species remains poorly understood; yet, these animals are immersed in and constantly interacting with a dynamic ocean. In this thesis, I developed a model framework that leverages 3D oceanographic data to improve geolocation error and demonstrated its utility for two species that, historically, have proven challenging to geolocate. This method is poised to leverage future advances as oceanographic data and models continue to improve.

Using the approach I developed and validated in \cref{chap:2}, I applied the model in \cref{chap:3} to study basking sharks in the North Atlantic using a high-resolution oceanographic model. Sharks in this study were tagged around Cape Cod, MA and moved as far as SE Brazil, a total distance of > 17,000 km, and 59\% of tracked individuals exhibited seasonal fidelity to the northeastern US shelf. During these movements, basking sharks made impressive vertical movements to >1,500 m and regularly occupied the mesopelagic, often for months at a time. This study draws attention to the poorly understood ecology of basking sharks and demonstrates the need for international cooperation in managing this highly migratory species that traverses the jurisdictions of dozens of countries each year.

I also applied the model framework from \cref{chap:2} to study the broadbill swordfish (\cref{chap:4}). Due to diel vertical migration and often aphotic behavior, the swordfish has historically proven difficult to study with traditional light-based geolocation. By leveraging significant improvements in geolocation error, I was able to describe mesoscale eddy use by some satellite-tagged individuals in this study. I validated the tag-based observations of eddy use by integrating 2 large fisheries datasets with the independent satellite tag data. Overall, I found swordfish exhibited extreme physiological versatility by occupying a 25$^\circ$C SST range; however, satellite-tagging and fisheries data independently suggested swordfish preferred a much narrower SST range in oligotrophic waters with active front boundaries. Overall, this study makes significant contributions to swordfish ecology and fisheries oceanography by quantifying some of the critical oceanographic influences on swordfish.

In the final data chapter (Chapter \ref{chap:5}), I used the "lessons learned" from the limitations of the previous chapters to design a tagging study to specifically quantify oceanographic associations of blue sharks, as a model pelagic predator. I focused this analysis on the eddy-rich region of the Gulf Stream where I was able to observe blue shark movements and behavior in high-resolution, 3-D as they moved through the eddy field. Using this data, I discovered that blue sharks actively sought the interiors of anticyclonic Gulf Stream eddies. These "warm core rings" are traditionally characterized as low productivity and anomalously warm, "desert-like" environments. Yet, blue sharks exhibited behaviors indicative of foraging in these features. With high-resolution dive data, I found blue sharks were regularly diving into the mesopelagic during the day in anticyclones, suggesting warm water at depth in these features may alleviate thermal constraints to diving into the mesopelagic to forage. This study provides important insight into the connectivity between epi- and mesopelagic ecosystems and suggests mesopelagic fishes may be a key link between planktonic production and top predators.

%----------------

\section{Outlook}
% limitations/next steps
As with any scientific endeavor, many assumptions were necessary in each of these studies and all left me with more fascinating questions to pursue. The modelling approach developed in \cref{chap:2} was an effective first step at building a flexible, transferable framework that can be readily adapted to animal movement problems in a wide range of taxa and environments. However, many improvements could be implemented, including several provided as feedback from model users. The primary limitations of \texttt{HMMoce} in its original form were its reliance on deprecated, manufacturer-specific software for light-based position estimates as well as support for tags built by a single manufacturer. This was appropriate for our needs, but it was apparent that additional compatibility would be required for widespread adoption of these methods. Further development to incorporate some of these changes are currently underway. In addition, additional process models (\eg including advection) and algorithm support (\eg Viterbi) will dramatically improve utility of this model for the broader community.

While the approach in \cref{chap:3} demonstrated remarkable improvements over previous studies to geolocate a traditionally challenging species, the resulting study was a primarily qualitative depiction of basking shark ecology in the NWA. Future studies could use this or similar datasets to better quantify oceanographic associations, particularly using the high-resolution dive data collected from many of these deployments. In the swordfish analysis in \cref{chap:4}, I was able to quantify habitat use and relate this to oceanography; however, this work would also benefit from further exploration of fine-scale behaviors using the depth-temperature time series from these tags.

Finally, \cref{chap:5} was my attempt to leverage "lessons learned" from the previous chapters to design, fund and conduct a study from end-to-end that would not be constrained by the analytical limitations I faced in previous chapters. While I believe this was largely successful, the high individual variability in eddy use calls for a use-availability control on this data to better quantify blue shark eddy use and identify potential drivers of this behavior. This study would also benefit from incorporating the role of underlying currents, as blue sharks are known to be heavily influenced by advection \citep[\eg][]{Carey1990} and from quantifying fine-scale response to gradients indicative of eddy encounter (\eg sea surface height). These, and other analyses, may aid in further interpretation of the mechanisms driving response of pelagic fishes to mesoscale eddies and their potential functional role(s).



\begin{comment}
%% set the stage: broadly describe in P1 the context within which your thesis operates
%Climate change is arguably one of the most urgent issues presently confronting the world. With no end in sight to the anthropogenic emission of carbon dioxide, expedited warming, sea level rise, and shifts in local and global weather patterns are likely. These changes, however, are not limited to terrestrial ecosystems as coordinated changes in the temperature, carbonate chemistry, stratification, and nutrient environment of the ocean occur. These rapid changes to the Earth's ecosystem highlight the importance of a holistic understanding of the global carbon cycle in which phytoplankton are key players. These climate-related changes to marine ecosystems will likely change the biogeography of phytoplankton as organisms adapt to new environmental conditions. Predicting these changes to the phytoplankton distributions hinges upon understanding the response of phytoplankton both at the community and species-level. In this thesis, I presented new tools (Chapter \ref{chap:2}) and approaches (Chapter \ref{chap:3}-\ref{chap:5}) to survey, characterize, and quantify the response of phytoplankton to changes in their nutrient environment \textit{in situ}. \par

%% what does your thesis do? typically you provide ~1 sentence summary of each chapter within a slightly broader narrative of "what it all means". this may be a few paragraphs.
%This thesis first introduces the development of a statistical approach to identify stably expressed genes for use both in qRT-PCR and metatranscriptomic studies from high-throughput sequencing datasets (Chapter \ref{chap:2}). Many of the ``housekeeping genes'' that are commonly used as references in qRT-PCR studies of phytoplankton were found to be variably expressed across the conditions used in this study. Such variability in a reference gene can drastically alter the gene expression patterns of the gene of interest. This approach broadens the pool of possible gene targets by facilitating the selection of genes without \textit{a priori} knowledge of function. Application of this technique to the metatranscriptomic study in Chapter \ref{chap:3} identified unique stable reference genes for both of the dominant diatoms and facilitated the quantitative comparison of their transcriptional signals. 

%The field study in Narragansett Bay, presented in Chapter \ref{chap:3}, approached the problem of assessing and comparing diatom nutrient physiology in two ways. Metabolic pathways with known relationships to N and P metabolism were tracked for both taxa. This approach suggested that the more dominant \textit{Skeletonema} was likely utilizing inorganic nutrients, where as the less abundant \textit{T. rotula} was relying upon organic compounds, like amino acids. This study, however, went further by combining field samples with incubation experiments. Using a Bayesian approach, these incubation experiments were used to 1) identify novel resource responsive gene targets and 2) contextualize expression signals to directly compare the physiology of organisms \textit{in situ}. This second step was accomplished by proportionalizing the signals in the field to the metabolic bounds on expression observed in the incubation experiments. These data demonstrated significant differences in the expression patterns of resource responsive genes, indicating that these two species are responding differently to the same nutrient environment. Narragansett Bay is an ideal model system for investigating the interplay between diversity and biogeochemistry as well as basic questions in ecology, such as the role of specialization in a dynamic ecosystem. Given the opportunity, I would want to sample this community throughout the year, in combination with similar nutrient experiments that were run for longer duration in larger volumes. While this technique was designed to examine the response of phytoplankton to nutrient dynamics, it could quite easily be expanded to other systems or variables. 

%Chapters \ref{chap:4} and \ref{chap:5}, both set in the oligotrophic North Pacific Subtropical Gyre, look to the opposite ends of the diversity spectrum, assessing the differences in response to nutrient pulse amongst functional groups and amongst strains. By sampling and comparing the global gene expression of the eukaryotic community both \textit{in situ} and in simulated deep seawater (DSW) upwelling incubations, the specific drivers of production for taxonomic groups were identified (Chapter \ref{chap:4}). The divergence in the variable transcript allocation ratio (VTAR) between the diatoms and haptophytes was one of the more surprising outcomes of this study, as it suggests that the r- and K-strategies as defined by enzyme kinetics and growth rate appear to hold true at the level of the transcription. I am quite interested to see if this schism in diatom and haptophyte VTAR is consistent across other stimuli, timescales, and systems\footnote{A preliminary glance at the data from Narragansett Bay suggests that this difference may be upheld.}. In the near future, I would like to replicate this study, with an additional sample 24-hours post-DSW addition in order to assess the more immediate response of the community to nutrient addition, potentially removing the signals associated with shifts in the community structure. 

%The other rather perplexing outcome from this study was the lack of response seen in the dinoflagellates following DSW addition. Chapter \ref{chap:4} suggests that this might be due to the fact that dinoflagellates might have alternative trophic strategies or live symbiotically. However, I suspect that that it is more of a general characteristic of dinoflagellate transcriptional regulation, or, rather, the lack thereof. It is known that autotrophic dinoflagellates do respond at the protein and activity level to changes in their nutrient environment \citep{Dyhrman1999}. Though not discussed in this thesis, expression patterns of \textit{Prorocentrum} in the Narragansett Bay incubations suggest that this pattern of no regulation may hold true in coastal dinoflagellates as well. Together, these data seem to indicate that metabolic plasticity in the dinoflagellates is happening at the protein level. It remains to be seen if these general patterns are upheld in future transcriptomic and metatranscripomic studies. 

%Finally, Chapter \ref{chap:5} investigates the intersecting roles of metabolic plasticity and strain diversity in the response of the \textit{Emiliania huxleyi} species complex to changing nutrient environments. The co-existence of at least five strains was observed in the environment, as defined by the variable gene sets. Additionally, following the addition of N, the strain composition of the field appeared to be altered. These findings support the hypothesis of \citet{Read2013} that the variability in gene content as observed in the pan genome is present and modulated in the field. Additionally, metabolic plasticity, as defined by changes in transcript abundance, was observed, particularly in metabolic processes associated with N and P metabolism, as well as life stage and calcification. The fact that any genes were identified as significantly differentially regulated in this study is surprising, as the two ``biological replicates'' used were actually distinct incubation experiments that were performed approximately two weeks apart with different initial populations. This, combined with the fact that many of these gene targets were also identified as significantly differentially regulated in N and P limited cultures in proteomic work by \citet{McKew2015}, suggests that these responses to nutrient stress are highly conserved within the \textit{E. huxleyi} species consortium. In my opinion this study highlights the importance of culturing diversity. In many ways, even with metatranscriptome assembly, we are limited by what we know; our ability to assign function or taxonomy to meta-omic data in the field hinges upon our databases. 

%% in summary, what have we learned from this thesis? what challenges/questions remain?
%Together these data provide new insights into the inner workings of the phytoplankton consortia, illuminating potentially missed co-limitations, capabilities, and interactions. Yet, the ultimate fate of these data beyond my very narrow field remains uncertain. Currently there is a disconnect between the type of data that we biologists want to collect and the type of data that ecosystem and ocean modelers want. The undertone of some of the chapters in this thesis is that these findings might be integrated into biogeochemical models. While I do believe that progress is being made toward this ultimate goal\footnote{In particular, the Darwin Project \citep{Follows2007} and some interesting, as of yet unpublished, work by Victoria Coles are pushing the boundaries on this integration.}, many hurdles remain \citep{Hood2007, Worden2015}. With continued collaboration and discourse between molecular ecologists and ecosystem modelers, I am hopeful that a solid quantitative connection might be made. 



\end{comment}