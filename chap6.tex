\chapter{Conclusion and Outlook}
\label{chap:6}
\clearpage
\raggedbottom

\textit{``We cannot cheat on DNA. We cannot get round photosynthesis. We cannot say I am not going to give a damn about phytoplankton. All these tiny mechanisms provide the preconditions of our planetary life.'' }
\begin{flushright}-- Barbara Ward, \textit{Who Speaks for Earth?}\footnote{Barbara Ward et al. Edited by Maurice F. Strong. (1973) \textit{Who Speaks for Earth?} New York, Norton. }\end{flushright}
    Climate change is arguably one of the most urgent issues presently confronting the world. With no end in sight to the anthropogenic emission of carbon dioxide, expedited warming, sea level rise, and shifts in local and global weather patterns are likely. These changes, however, are not limited to terrestrial ecosystems, as coordinated changes in the temperature, carbonate chemistry, stratification, and nutrient environment of the ocean occur. These rapid changes to the Earth's ecosystem highlight the importance of a holistic understanding of the global carbon cycle, in which phytoplankton are key players. These climate-related changes to marine ecosystems will likely change the biogeography of phytoplankton, as organisms adapt to new environmental conditions. Predicting these changes to the phytoplankton distributions hinges upon understanding the response of phytoplankton both at the community and species-level. In this thesis, I presented new tools (Chapter \ref{chap:2}) and approaches (Chapter \ref{chap:3}-\ref{chap:5}) to survey, characterize, and quantify the response of phytoplankton to changes in their nutrient environment \textit{in situ}. 

\par
\section{Summary}
In this thesis, I first discuss the development of an approach for use in identifying stably expressed genes for use both in qRT-PCR and metatranscriptomic studies through the statistical parsing of high-through put sequence data (Chapter \ref{chap:2}). Many of the ``housekeeping genes'' that are commonly used as references in qRT-PCR studies of phytoplankton were found to be variably expressed. Such changes in a reference gene can drastically alter the interpretation of gene expression patterns in one's gene of interest. %Ass this approach allows the selection and identification of stably expressed genes, without \textit{a priori} knowledge of function, .


As the set of Thus, we can move beyond the set of frequently used ``housekeeping'' genes, many of which were found to not be stably expressed in this study, 
moleculesconnection to the immense oceanic environment

dove deeply into the molecular physiology and ecology of individual tiny cells living with in a consortium of other tiny cells  

unify an idea that I have been struggling with throughout my PhD

This quote by Barbara Ward

\section{Finding a place for 'omics in a modeling world}

During my PhD it seems that one of the largest hurdles that I have come up against has been trying to figure out how to explain the significance of my work to ocean modelers. The undertone of some of the chapters in this thesis (e.g. Chapter \ref{4}) was that these findings might be easily integrated into biogeochemical models. A

Many of the chapters in this thesis were laced with undertones suggesting that the results from these studies might be easily integrated into biogeochemical models. 

While I am not saying that I think this is a hopeless 

\section{Future Directions}


The ‘omic revolution has shifted the limiting step in understanding the microbial role in biogeochemical cycles away from difficulty in directly monitoring the metabolism of microbial communities to data integration. We must be able to quantitatively unify both physical and chemical environmental data with large ‘omic’ datasets of all varieties (from genomic to metabolomic) if we wish to address larger questions of ecology. This is a computational and quantitative problem that is of great interest to me as I believe it will allow us not only to better address overarching questions, but that cross use of these data will inform the analysis and interpretation of each individual data type. 


\section{Conclusions and future directions}
Capitalizing upon this revolution to address biogeochemical or ecological questions, however, is becoming increasingly difficult, as our ability to make measurements has surpassed our ability to analyze, visualize, and compare the data produced.



Barbara Ward, a British economist during the 20\textsuperscript{th} century, was an early, some might say before her time, proponent of sustainable development. She possessed, what I think was, a very holistic view of the intersection of economic and environmental concerns, arguing strongly for humans to practice ``careful husbandry'' of the Earth. Though slightly crass, this quote, aside from hitting somewhat coincidentally upon the major objects of my thesis (phytoplankton, DNA...), neatly sums up a point that I have been struggling with during my PhD. For my research, I dove into the smallest detail, tracking the shifts in RNA composition of a consortium of very tiny organisms living within a very small parcel of water. After analyzing these data, it was hard to figure out a way to connect the results to the global system. These findings at the sub-micro scale seem to detached from the global system to have any real import. But, as Barbara Ward points out, these ``tiny mechanisms provide the conditions preconditions of \textit{our} planetary life.'' 
