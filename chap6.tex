

\chapter{Summary and Conclusions}
\label{chap:6}

\raggedbottom

\textit{``We cannot cheat on DNA. We cannot get round photosynthesis. We cannot say I am not going to give a damn about phytoplankton. All these tiny mechanisms provide the preconditions of our planetary life.'' }

\begin{flushright}-- Barbara Ward, \textit{Who Speaks for Earth?}\footnote{Barbara Ward et al. Edited by Maurice F. Strong. (1973) \textit{Who Speaks for Earth?} New York, Norton. }\end{flushright}



taking these data, I extrapolated what these findings meant to an ecosystem or even global level. In many ways, this sort of extrapolation is incredibly uncomfortable, but 

moleculesconnection to the immense oceanic environment

dove deeply into the molecular physiology and ecology of individual tiny cells living with in a consortium of other tiny cells  

unify an idea that I have been struggling with throughout my PhD

This quote by Barbara Ward

\section{Finding a place for 'omics in a modeling world}

During my PhD it seems that one of the largest hurdles that I have come up against has been trying to figure out how to explain the significance of my work to ocean modelers. The undertone of some of the chapters in this thesis (e.g. Chapter \ref{4}) was that these findings might be easily integrated into biogeochemical models. A

Many of the chapters in this thesis were laced with undertones suggesting that the results from these studies might be easily integrated into biogeochemical models. 

While I am not saying that I think this is a hopeless 

\section{Future Directions}

\section{Conclusions and future directions}
Capitalizing upon this revolution to address biogeochemical or ecological questions, however, is becoming increasingly difficult, as our ability to make measurements has surpassed our ability to analyze, visualize, and compare the data produced.



Barbara Ward, a British economist during the 20\textsuperscript{th} century, was an early, some might say before her time, proponent of sustainable development. She possessed, what I think was, a very holistic view of the intersection of economic and environmental concerns, arguing strongly for humans to practice ``careful husbandry'' of the Earth. Though slightly crass, this quote, aside from hitting somewhat coincidentally upon the major objects of my thesis (phytoplankton, DNA...), neatly sums up a point that I have been struggling with during my PhD. For my research, I dove into the smallest detail, tracking the shifts in RNA composition of a consortium of very tiny organisms living within a very small parcel of water. After analyzing these data, it was hard to figure out a way to connect the results to the global system. These findings at the sub-micro scale seem to detached from the global system to have any real import. But, as Barbara Ward points out, these ``tiny mechanisms provide the conditions preconditions of \textit{our} planetary life.'' 
